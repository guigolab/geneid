% -*- mode: Noweb; noweb-code-mode: perl-mode -*-% ===> this file was generated automatically by noweave --- better not edit it
\documentclass[10pt]{article}
% 
% I'm trying to learn Literate Programming. I will use sgp2 as an LP exercise.
%
% The same document contains the code which will be generated with the proper filter:
%
%   pqtangle -L sgp2.tex
%
% The project documentation is also available from the same file:
%
%   latex sgp2.tex; dvips sgp2.dvi -o sgp2.ps  # to obtain the PS version 
%
%   pdflatex sgp2.tex;                         # to obtain the PS version
%
% $Id: sgp2.tex,v 1.3 2000-10-10 20:07:52 jabril Exp $
%  
\usepackage{noweb,multicol}
\usepackage[offset={0pt,0pt},hmargin={2.5cm,2cm},vmargin={1cm,0.25cm}]{geometry}
\noweboptions{smallcode}
% relax page breaking scheme
\def\nwendcode{\endtrivlist \endgroup}
\let\nwdocspar=\par

\newcommand{\sctn}[1]{\section{#1}}
\newcommand{\subsctn}[1]{\subsection{#1}}
\newcommand{\desc}[1]{\item[#1] \ \\}

\newcommand{\cd}[1]{\texttt{\textbf{#1}}} % program<code> font style definition
\newcommand{\cv}[1]{\texttt{\textsl{#1}}} % program<vars> font style definition
\newcommand{\cp}[1]{\textit{#1}} % program<vars> font style definition

\newcommand{\sgp}{{\Tt{}sgp2}} % PROGRAM alias

% TITLE
\title{Sintenic Gene Prediction Tool}
\author{Josep F. Abril\\Roderic Guig\'o} %'

\begin{document}

\maketitle

\begin{center}
{\Large\sgp}\\[1ex]
{\large\verb+$Id: sgp2.tex,v 1.3 2000-10-10 20:07:52 jabril Exp $+}\\[2ex]
\end{center}

\begin{abstract}
The initial goal for Josep F. Abril will be to implement a \cd{perl} version of \sgp{}, which works like the current shell implementation. Secondary goals are to introduce myself into CVS (Concurrent Version Control) project management and Literate Programming (this document will serve as example).
\end{abstract}

\sctn{Program Basic Outline}

Our initial program outline (our root chunk) looks like:\\[1ex]

\nwfilename{./sgp2.nw}\nwbegincode{1}\sublabel{NWsgp9-sgp7-1}\nwmargintag{{\nwtagstyle{}\subpageref{NWsgp9-sgp7-1}}}\moddef{sgp2.pl~{\nwtagstyle{}\subpageref{NWsgp9-sgp7-1}}}\endmoddef\nwstartdeflinemarkup\nwenddeflinemarkup
\LA{}shebang~{\nwtagstyle{}\subpageref{NWsgp9-she7-1}}\RA{}
\LA{}version control~{\nwtagstyle{}\subpageref{NWsgp9-verF-1}}\RA{}
\LA{}declare uses~{\nwtagstyle{}\subpageref{NWsgp9-decC-1}}\RA{}

### VARS ###

\LA{}declare variables~{\nwtagstyle{}\subpageref{NWsgp9-decH-1}}\RA{}
\LA{}define defaults~{\nwtagstyle{}\subpageref{NWsgp9-defF-1}}\RA{}

### SUBS ###

\LA{}global subroutines~{\nwtagstyle{}\subpageref{NWsgp9-gloI-1}}\RA{}
\LA{}get options~{\nwtagstyle{}\subpageref{NWsgp9-getB-1}}\RA{}

### MAIN SUBS ###

\LA{}run blast~{\nwtagstyle{}\subpageref{NWsgp9-run9-1}}\RA{}
\LA{}extract HSPs~{\nwtagstyle{}\subpageref{NWsgp9-extC-1}}\RA{}
\LA{}run geneid~{\nwtagstyle{}\subpageref{NWsgp9-runA-1}}\RA{}
\LA{}graphical output~{\nwtagstyle{}\subpageref{NWsgp9-graG-1}}\RA{}

### MAIN LOOP ###

\LA{}Main Loop~{\nwtagstyle{}\subpageref{NWsgp9-Mai9-1}}\RA{}

### EOF ###

\LA{}POD man page trailer~{\nwtagstyle{}\subpageref{NWsgp9-PODK-1}}\RA{}
\nwnotused{sgp2.pl}\nwendcode{}\nwbegindocs{2}% 

\nwenddocs{}\nwbegincode{3}\sublabel{NWsgp9-she7-1}\nwmargintag{{\nwtagstyle{}\subpageref{NWsgp9-she7-1}}}\moddef{shebang~{\nwtagstyle{}\subpageref{NWsgp9-she7-1}}}\endmoddef\nwstartdeflinemarkup\nwusesondefline{\\{NWsgp9-sgp7-1}}\nwenddeflinemarkup
#!/usr/bin/perl -w
my $Start = time;
\nwused{\\{NWsgp9-sgp7-1}}\nwendcode{}\nwbegindocs{4} %def $Start

'\cd{perl -w}' prints all sorts of useful and interesting warning messages at compile time.

{\Tt{}{\$}Start} %$
contains program starting time (seconds since \textsc{Unix} ``epoch''), which will be used by timing function to calculate for how long was running our program.

\nwenddocs{}\nwbegincode{5}\sublabel{NWsgp9-verF-1}\nwmargintag{{\nwtagstyle{}\subpageref{NWsgp9-verF-1}}}\moddef{version control~{\nwtagstyle{}\subpageref{NWsgp9-verF-1}}}\endmoddef\nwstartdeflinemarkup\nwusesondefline{\\{NWsgp9-sgp7-1}}\nwenddeflinemarkup
my $PROGRAM = "sgp2";
my @tmp_ver = split / +/, ' $Id: sgp2.tex,v 1.3 2000-10-10 20:07:52 jabril Exp $ ';
my $VERSION = "v$tmp_ver[3] [$tmp_ver[4] $tmp_ver[5] $tmp_ver[7]]";
\nwindexdefn{\nwixident{{\$}PROGRAM}}{:doPROGRAM}{NWsgp9-verF-1}\nwindexdefn{\nwixident{{\$}VERSION}}{:doVERSION}{NWsgp9-verF-1}\nwindexdefn{\nwixident{@tmp{\_}ver}}{@tmp:unver}{NWsgp9-verF-1}\eatline
\nwused{\\{NWsgp9-sgp7-1}}\nwidentdefs{\\{{\nwixident{{\$}PROGRAM}}{:doPROGRAM}}\\{{\nwixident{@tmp{\_}ver}}{@tmp:unver}}\\{{\nwixident{{\$}VERSION}}{:doVERSION}}}\nwendcode{}\nwbegindocs{6}\nwdocspar
Take into account that here is shown {\Tt{}sgp2.nw} versions, but once the perl script is extracted and commited then it follows {\Tt{}sgp2.pl} version numbering.

\nwenddocs{}\nwbegincode{7}\sublabel{NWsgp9-decC-1}\nwmargintag{{\nwtagstyle{}\subpageref{NWsgp9-decC-1}}}\moddef{declare uses~{\nwtagstyle{}\subpageref{NWsgp9-decC-1}}}\endmoddef\nwstartdeflinemarkup\nwusesondefline{\\{NWsgp9-sgp7-1}}\nwprevnextdefs{\relax}{NWsgp9-decC-2}\nwenddeflinemarkup
use strict;
\nwalsodefined{\\{NWsgp9-decC-2}\\{NWsgp9-decC-3}\\{NWsgp9-decC-4}}\nwused{\\{NWsgp9-sgp7-1}}\nwendcode{}\nwbegindocs{8}\nwdocspar

'\cd{use strict;}' restricts unsafe constructs, like attempting to use missed symbolic references ('\cv{refs}'), undeclared variables ('\cv{vars}') or not predeclared subroutine ('\cv{subs}').

\sctn{Main Program Loop}

\nwenddocs{}\nwbegincode{9}\sublabel{NWsgp9-Mai9-1}\nwmargintag{{\nwtagstyle{}\subpageref{NWsgp9-Mai9-1}}}\moddef{Main Loop~{\nwtagstyle{}\subpageref{NWsgp9-Mai9-1}}}\endmoddef\nwstartdeflinemarkup\nwusesondefline{\\{NWsgp9-sgp7-1}}\nwenddeflinemarkup
&Which_Options();

&Run_Blast()   if $blast_flg;

&Extract_HSP() if $hsps_flg;

&Run_geneid()  if $geneid_flg;

&Make_Plots()  if $plots_flg; 

&get_exec_time(time);

exit(1);
\nwused{\\{NWsgp9-sgp7-1}}\nwidentuses{\\{{\nwixident{{\&}Extract{\_}HSP}}{:amExtract:unHSP}}\\{{\nwixident{{\&}get{\_}exec{\_}time}}{:amget:unexec:untime}}\\{{\nwixident{{\&}Make{\_}Plots}}{:amMake:unPlots}}\\{{\nwixident{{\&}Run{\_}Blast}}{:amRun:unBlast}}\\{{\nwixident{{\&}Run{\_}geneid}}{:amRun:ungeneid}}\\{{\nwixident{{\&}Which{\_}Options}}{:amWhich:unOptions}}}\nwindexuse{\nwixident{{\&}Extract{\_}HSP}}{:amExtract:unHSP}{NWsgp9-Mai9-1}\nwindexuse{\nwixident{{\&}get{\_}exec{\_}time}}{:amget:unexec:untime}{NWsgp9-Mai9-1}\nwindexuse{\nwixident{{\&}Make{\_}Plots}}{:amMake:unPlots}{NWsgp9-Mai9-1}\nwindexuse{\nwixident{{\&}Run{\_}Blast}}{:amRun:unBlast}{NWsgp9-Mai9-1}\nwindexuse{\nwixident{{\&}Run{\_}geneid}}{:amRun:ungeneid}{NWsgp9-Mai9-1}\nwindexuse{\nwixident{{\&}Which{\_}Options}}{:amWhich:unOptions}{NWsgp9-Mai9-1}\nwendcode{}\nwbegindocs{10}\nwdocspar

\subsctn{Default values setting}

\nwenddocs{}\nwbegincode{11}\sublabel{NWsgp9-defF-1}\nwmargintag{{\nwtagstyle{}\subpageref{NWsgp9-defF-1}}}\moddef{define defaults~{\nwtagstyle{}\subpageref{NWsgp9-defF-1}}}\endmoddef\nwstartdeflinemarkup\nwusesondefline{\\{NWsgp9-sgp7-1}}\nwenddeflinemarkup
my $NF = 0; # number of input files
my $SGP2 = ''; # main path to SGP2 binaries, scripts and params

\nwindexdefn{\nwixident{{\$}NF}}{:doNF}{NWsgp9-defF-1}\nwindexdefn{\nwixident{{\$}SGP2}}{:doSGP2}{NWsgp9-defF-1}\eatline
\nwused{\\{NWsgp9-sgp7-1}}\nwidentdefs{\\{{\nwixident{{\$}NF}}{:doNF}}\\{{\nwixident{{\$}SGP2}}{:doSGP2}}}\nwendcode{}\nwbegindocs{12}\nwdocspar
\subsctn{Parsing Command-line Options}

\nwenddocs{}\nwbegincode{13}\sublabel{NWsgp9-decC-2}\nwmargintag{{\nwtagstyle{}\subpageref{NWsgp9-decC-2}}}\moddef{declare uses~{\nwtagstyle{}\subpageref{NWsgp9-decC-1}}}\plusendmoddef\nwstartdeflinemarkup\nwusesondefline{\\{NWsgp9-sgp7-1}}\nwprevnextdefs{NWsgp9-decC-1}{NWsgp9-decC-3}\nwenddeflinemarkup
use Getopt::Long;
Getopt::Long::Configure qw/ bundling pass_through /;
\nwused{\\{NWsgp9-sgp7-1}}\nwendcode{}\nwbegindocs{14}\nwdocspar

\begin{description}
\desc{\cd{use Getopt::Long;}} This module implements an extended getopt function called \cd{GetOptions()}. This function adheres to the POSIX syntax for command line options, with GNU extensions. In general, this means that options having long names instead of single letters are introduced with a double dash "\-\-". 

\desc{\cd{Getopt::Long::Configure qw(} \cv{bundling} \cv{pass\_through} \cd{);}} \cd{GetOptions} can be configured by calling subroutine \cd{Getopt::Long::Configure}. This subroutine takes a list of quoted strings, each specifying a configuration option to be set. Options can be reset by prefixing with \cv{no\_}. 

 \begin{description}
 \desc{\cv{bundling}} Support for bundling of command line options, as was the case with the more traditional single-letter approach (introduced with a single dash "-"), is provided but not enabled by default. 

 \desc{\cv{pass\_through}} Unknown options are passed through in \cd{@ARGV} instead of being flagged as errors. This makes it possible to write wrapper scripts that process only part of the user supplied options, and passes the remaining options to some other program.
 \end{description}
\end{description}

\nwenddocs{}\nwbegincode{15}\sublabel{NWsgp9-decH-1}\nwmargintag{{\nwtagstyle{}\subpageref{NWsgp9-decH-1}}}\moddef{declare variables~{\nwtagstyle{}\subpageref{NWsgp9-decH-1}}}\endmoddef\nwstartdeflinemarkup\nwusesondefline{\\{NWsgp9-sgp7-1}}\nwprevnextdefs{\relax}{NWsgp9-decH-2}\nwenddeflinemarkup
# GetOptions Variables
my ( $seq1,$seq2,$geneid_opt,$geneid_param,$blast_opt,$score_cutoff,
     $shrink,$tbx,$hsp,$ofn,$ps_output,$verbose_flg,$help_flg );
\nwindexdefn{\nwixident{{\$}seq1}}{:doseq1}{NWsgp9-decH-1}\nwindexdefn{\nwixident{{\$}seq2}}{:doseq2}{NWsgp9-decH-1}\nwindexdefn{\nwixident{{\$}geneid{\_}opt}}{:dogeneid:unopt}{NWsgp9-decH-1}\nwindexdefn{\nwixident{{\$}geneid{\_}param}}{:dogeneid:unparam}{NWsgp9-decH-1}\nwindexdefn{\nwixident{{\$}blast{\_}opt}}{:doblast:unopt}{NWsgp9-decH-1}\nwindexdefn{\nwixident{{\$}score{\_}cutoff}}{:doscore:uncutoff}{NWsgp9-decH-1}\nwindexdefn{\nwixident{{\$}shrink}}{:doshrink}{NWsgp9-decH-1}\nwindexdefn{\nwixident{{\$}tbx}}{:dotbx}{NWsgp9-decH-1}\nwindexdefn{\nwixident{{\$}hsp}}{:dohsp}{NWsgp9-decH-1}\nwindexdefn{\nwixident{{\$}ofn}}{:doofn}{NWsgp9-decH-1}\nwindexdefn{\nwixident{{\$}ps{\_}output}}{:dops:unoutput}{NWsgp9-decH-1}\nwindexdefn{\nwixident{{\$}verbose{\_}flg}}{:doverbose:unflg}{NWsgp9-decH-1}\nwindexdefn{\nwixident{{\$}help{\_}flg}}{:dohelp:unflg}{NWsgp9-decH-1}\eatline
\nwalsodefined{\\{NWsgp9-decH-2}}\nwused{\\{NWsgp9-sgp7-1}}\nwidentdefs{\\{{\nwixident{{\$}blast{\_}opt}}{:doblast:unopt}}\\{{\nwixident{{\$}geneid{\_}opt}}{:dogeneid:unopt}}\\{{\nwixident{{\$}geneid{\_}param}}{:dogeneid:unparam}}\\{{\nwixident{{\$}help{\_}flg}}{:dohelp:unflg}}\\{{\nwixident{{\$}hsp}}{:dohsp}}\\{{\nwixident{{\$}ofn}}{:doofn}}\\{{\nwixident{{\$}ps{\_}output}}{:dops:unoutput}}\\{{\nwixident{{\$}score{\_}cutoff}}{:doscore:uncutoff}}\\{{\nwixident{{\$}seq1}}{:doseq1}}\\{{\nwixident{{\$}seq2}}{:doseq2}}\\{{\nwixident{{\$}shrink}}{:doshrink}}\\{{\nwixident{{\$}tbx}}{:dotbx}}\\{{\nwixident{{\$}verbose{\_}flg}}{:doverbose:unflg}}}\nwendcode{}\nwbegindocs{16}%$

\nwenddocs{}\nwbegincode{17}\sublabel{NWsgp9-getB-1}\nwmargintag{{\nwtagstyle{}\subpageref{NWsgp9-getB-1}}}\moddef{get options~{\nwtagstyle{}\subpageref{NWsgp9-getB-1}}}\endmoddef\nwstartdeflinemarkup\nwusesondefline{\\{NWsgp9-sgp7-1}}\nwenddeflinemarkup
sub Which_Options() \{
    GetOptions( 
                "1"        => \\$seq1         , # seqfile_1
                "2"        => \\$seq2         , # seqfile_2
                "g"        => \\$geneid_opt   , # geneid options      
                "P"        => \\$geneid_param , # geneid parameter file 
                "o"        => \\$blast_opt    , # tblastx options 
                "c"        => \\$score_cutoff , # tblastx score cutoff 
                "s"        => \\$shrink       , # shrink hsp's by
                "t"        => \\$tbx          , # read tblastx from file
                "f"        => \\$hsp          , # read HSP files in directory
                "k"        => \\$ofn          , # intermediate filename
                "p"        => \\$ps_output    , # postscript output 
                "v"        => \\$verbose_flg  , # verbose    
                "h|help|?" => \\$help_flg     , # print help
                );
    &prt_Help if $help_flg;
\}; # sub Which_Options
\nwindexdefn{\nwixident{{\&}Which{\_}Options}}{:amWhich:unOptions}{NWsgp9-getB-1}\eatline
\nwused{\\{NWsgp9-sgp7-1}}\nwidentdefs{\\{{\nwixident{{\&}Which{\_}Options}}{:amWhich:unOptions}}}\nwidentuses{\\{{\nwixident{{\$}blast{\_}opt}}{:doblast:unopt}}\\{{\nwixident{{\$}geneid{\_}opt}}{:dogeneid:unopt}}\\{{\nwixident{{\$}geneid{\_}param}}{:dogeneid:unparam}}\\{{\nwixident{{\$}help{\_}flg}}{:dohelp:unflg}}\\{{\nwixident{{\$}hsp}}{:dohsp}}\\{{\nwixident{{\$}ofn}}{:doofn}}\\{{\nwixident{{\&}prt{\_}Help}}{:amprt:unHelp}}\\{{\nwixident{{\$}ps{\_}output}}{:dops:unoutput}}\\{{\nwixident{{\$}score{\_}cutoff}}{:doscore:uncutoff}}\\{{\nwixident{{\$}seq1}}{:doseq1}}\\{{\nwixident{{\$}seq2}}{:doseq2}}\\{{\nwixident{{\$}shrink}}{:doshrink}}\\{{\nwixident{{\$}tbx}}{:dotbx}}\\{{\nwixident{{\$}verbose{\_}flg}}{:doverbose:unflg}}}\nwindexuse{\nwixident{{\$}blast{\_}opt}}{:doblast:unopt}{NWsgp9-getB-1}\nwindexuse{\nwixident{{\$}geneid{\_}opt}}{:dogeneid:unopt}{NWsgp9-getB-1}\nwindexuse{\nwixident{{\$}geneid{\_}param}}{:dogeneid:unparam}{NWsgp9-getB-1}\nwindexuse{\nwixident{{\$}help{\_}flg}}{:dohelp:unflg}{NWsgp9-getB-1}\nwindexuse{\nwixident{{\$}hsp}}{:dohsp}{NWsgp9-getB-1}\nwindexuse{\nwixident{{\$}ofn}}{:doofn}{NWsgp9-getB-1}\nwindexuse{\nwixident{{\&}prt{\_}Help}}{:amprt:unHelp}{NWsgp9-getB-1}\nwindexuse{\nwixident{{\$}ps{\_}output}}{:dops:unoutput}{NWsgp9-getB-1}\nwindexuse{\nwixident{{\$}score{\_}cutoff}}{:doscore:uncutoff}{NWsgp9-getB-1}\nwindexuse{\nwixident{{\$}seq1}}{:doseq1}{NWsgp9-getB-1}\nwindexuse{\nwixident{{\$}seq2}}{:doseq2}{NWsgp9-getB-1}\nwindexuse{\nwixident{{\$}shrink}}{:doshrink}{NWsgp9-getB-1}\nwindexuse{\nwixident{{\$}tbx}}{:dotbx}{NWsgp9-getB-1}\nwindexuse{\nwixident{{\$}verbose{\_}flg}}{:doverbose:unflg}{NWsgp9-getB-1}\nwendcode{}\nwbegindocs{18}%$

\subsctn{Printing help: {\Tt{}prt{\_}Help()}}

\nwenddocs{}\nwbegincode{19}\sublabel{NWsgp9-decC-3}\nwmargintag{{\nwtagstyle{}\subpageref{NWsgp9-decC-3}}}\moddef{declare uses~{\nwtagstyle{}\subpageref{NWsgp9-decC-1}}}\plusendmoddef\nwstartdeflinemarkup\nwusesondefline{\\{NWsgp9-sgp7-1}}\nwprevnextdefs{NWsgp9-decC-2}{NWsgp9-decC-4}\nwenddeflinemarkup
use Pod::Text;
\nwused{\\{NWsgp9-sgp7-1}}\nwendcode{}\nwbegindocs{20}\nwdocspar

\cd{use Pod::Text;} This module converts POD data to formatted ASCII text. This defines the single function \cd{pod2text([\cv{-a},] [\cv{-width}\cp{N},] \cp{inputfile})}, that can take the following options: '\cv{-a}' for an alternative output format, '\cv{-width}' to set max terminal width. '\cp{inputfile}' must have POD format.

\nwenddocs{}\nwbegincode{21}\sublabel{NWsgp9-gloI-1}\nwmargintag{{\nwtagstyle{}\subpageref{NWsgp9-gloI-1}}}\moddef{global subroutines~{\nwtagstyle{}\subpageref{NWsgp9-gloI-1}}}\endmoddef\nwstartdeflinemarkup\nwusesondefline{\\{NWsgp9-sgp7-1}}\nwprevnextdefs{\relax}{NWsgp9-gloI-2}\nwenddeflinemarkup
# Prints help 
sub prt_Help() \{
    my $tmp_pod_file = "/tmp/sgp.pod";
    open(KI, "> $tmp_pod_file");
    while (<DATA>) \{
        s/\\$PROGRAM/$PROGRAM/g ;
        s/\\$VERSION/$tmp_ver[3]/g ;
        print KI $_ ;
    \};
    close(KI);
    pod2text($tmp_pod_file);
    unlink($tmp_pod_file) or die "Can't delete $tmp_pod_file: $!\\n";
    exit(1);
\} # sub prt_Help
\nwindexdefn{\nwixident{{\&}prt{\_}Help}}{:amprt:unHelp}{NWsgp9-gloI-1}\eatline
\nwalsodefined{\\{NWsgp9-gloI-2}\\{NWsgp9-gloI-3}}\nwused{\\{NWsgp9-sgp7-1}}\nwidentdefs{\\{{\nwixident{{\&}prt{\_}Help}}{:amprt:unHelp}}}\nwidentuses{\\{{\nwixident{{\$}PROGRAM}}{:doPROGRAM}}\\{{\nwixident{{\$}VERSION}}{:doVERSION}}}\nwindexuse{\nwixident{{\$}PROGRAM}}{:doPROGRAM}{NWsgp9-gloI-1}\nwindexuse{\nwixident{{\$}VERSION}}{:doVERSION}{NWsgp9-gloI-1}\nwendcode{}\nwbegindocs{22}%$

\cd{unlink()} This command deletes a list of files. Returns the number of files successfully deleted.

\subsctn{Some general subroutines}

\nwenddocs{}\nwbegincode{23}\sublabel{NWsgp9-decC-4}\nwmargintag{{\nwtagstyle{}\subpageref{NWsgp9-decC-4}}}\moddef{declare uses~{\nwtagstyle{}\subpageref{NWsgp9-decC-1}}}\plusendmoddef\nwstartdeflinemarkup\nwusesondefline{\\{NWsgp9-sgp7-1}}\nwprevnextdefs{NWsgp9-decC-3}{\relax}\nwenddeflinemarkup
\nwused{\\{NWsgp9-sgp7-1}}\nwendcode{}\nwbegincode{24}\sublabel{NWsgp9-decH-2}\nwmargintag{{\nwtagstyle{}\subpageref{NWsgp9-decH-2}}}\moddef{declare variables~{\nwtagstyle{}\subpageref{NWsgp9-decH-1}}}\plusendmoddef\nwstartdeflinemarkup\nwusesondefline{\\{NWsgp9-sgp7-1}}\nwprevnextdefs{NWsgp9-decH-1}{\relax}\nwenddeflinemarkup
\nwused{\\{NWsgp9-sgp7-1}}\nwendcode{}\nwbegincode{25}\sublabel{NWsgp9-gloI-2}\nwmargintag{{\nwtagstyle{}\subpageref{NWsgp9-gloI-2}}}\moddef{global subroutines~{\nwtagstyle{}\subpageref{NWsgp9-gloI-1}}}\plusendmoddef\nwstartdeflinemarkup\nwusesondefline{\\{NWsgp9-sgp7-1}}\nwprevnextdefs{NWsgp9-gloI-1}{NWsgp9-gloI-3}\nwenddeflinemarkup
# Reporting IN/OUT progress.
sub prt_progress() \{
    $verbose_flg && do \{
        print STDERR ".";
        (($_[0] % 50) == 0) && print STDERR "[".&fill_left($_[0],6,"0")."]\\n";
    \};
\} # END_SUB: prt_progress
#
sub prt_foeprg() \{
    $verbose_flg && ((($_[0] % 50) != 0) && print STDERR "[".&fill_left($_[0],6,"0")."]\\n" );
\} # END_SUB: prt_foeprg
# Get a fixed length string from a given string and filling char/s.
sub fill_right() \{ $_[0].($_[2] x ($_[1] - length($_[0]))) \}
sub fill_left()  \{ ($_[2] x ($_[1] - length($_[0]))).$_[0] \}
# returns the max value from input array
sub max() \{ my ($z) = shift @_; my $l; foreach $l (@_) \{ $z = $l if $l > $z ; \}; $z; \} 
\nwindexdefn{\nwixident{{\&}prt{\_}progress}}{:amprt:unprogress}{NWsgp9-gloI-2}\nwindexdefn{\nwixident{{\&}prt{\_}foeprg}}{:amprt:unfoeprg}{NWsgp9-gloI-2}\nwindexdefn{\nwixident{{\&}fill{\_}right}}{:amfill:unright}{NWsgp9-gloI-2}\nwindexdefn{\nwixident{{\&}fill{\_}left}}{:amfill:unleft}{NWsgp9-gloI-2}\nwindexdefn{\nwixident{{\&}max}}{:ammax}{NWsgp9-gloI-2}\eatline
\nwused{\\{NWsgp9-sgp7-1}}\nwidentdefs{\\{{\nwixident{{\&}fill{\_}left}}{:amfill:unleft}}\\{{\nwixident{{\&}fill{\_}right}}{:amfill:unright}}\\{{\nwixident{{\&}max}}{:ammax}}\\{{\nwixident{{\&}prt{\_}foeprg}}{:amprt:unfoeprg}}\\{{\nwixident{{\&}prt{\_}progress}}{:amprt:unprogress}}}\nwidentuses{\\{{\nwixident{{\$}verbose{\_}flg}}{:doverbose:unflg}}}\nwindexuse{\nwixident{{\$}verbose{\_}flg}}{:doverbose:unflg}{NWsgp9-gloI-2}\nwendcode{}\nwbegindocs{26}\nwdocspar
\begin{description}
\desc{\cv{prt\_progress()}}
\desc{\cv{prt\_foeprg()}}
\desc{\cv{fill\_right()}}
\desc{\cv{fill\_left()}}
\desc{\cv{max()}}
\end{description}

\nwenddocs{}\nwbegincode{27}\sublabel{NWsgp9-gloI-3}\nwmargintag{{\nwtagstyle{}\subpageref{NWsgp9-gloI-3}}}\moddef{global subroutines~{\nwtagstyle{}\subpageref{NWsgp9-gloI-1}}}\plusendmoddef\nwstartdeflinemarkup\nwusesondefline{\\{NWsgp9-sgp7-1}}\nwprevnextdefs{NWsgp9-gloI-2}{\relax}\nwenddeflinemarkup
# Timing.
sub get_exec_time() \{
    $verbose_flg && do \{
        my $End = $_[0];
        my ($c,$s,$m,$h,$r);
        $r = $End - $Start;
        $s = $r % 60;
        $r = ($r - $s) / 60;
        $m = $r % 60;
        $r = ($r - $m) / 60;
        $h = $r % 24;
        ($s,$m,$h) = (&fill_left($s,2,"0"),&fill_left($m,2,"0"),&fill_left($h,2,"0"));
print STDERR <<EOF;
##
##########################################################
## \\"$PROGRAM\\"  Execution Time:  $h:$m:$s
##########################################################
EOF
    \};
\} # END_SUB: get_exec_time
\nwindexdefn{\nwixident{{\&}get{\_}exec{\_}time}}{:amget:unexec:untime}{NWsgp9-gloI-3}\eatline
\nwused{\\{NWsgp9-sgp7-1}}\nwidentdefs{\\{{\nwixident{{\&}get{\_}exec{\_}time}}{:amget:unexec:untime}}}\nwidentuses{\\{{\nwixident{{\&}fill{\_}left}}{:amfill:unleft}}\\{{\nwixident{{\$}PROGRAM}}{:doPROGRAM}}\\{{\nwixident{{\$}verbose{\_}flg}}{:doverbose:unflg}}}\nwindexuse{\nwixident{{\&}fill{\_}left}}{:amfill:unleft}{NWsgp9-gloI-3}\nwindexuse{\nwixident{{\$}PROGRAM}}{:doPROGRAM}{NWsgp9-gloI-3}\nwindexuse{\nwixident{{\$}verbose{\_}flg}}{:doverbose:unflg}{NWsgp9-gloI-3}\nwendcode{}\nwbegindocs{28}%$

\cv{get\_exec\_time()}

\sctn{Running {\Tt{}blast} on sequences}

\nwenddocs{}\nwbegincode{29}\sublabel{NWsgp9-run9-1}\nwmargintag{{\nwtagstyle{}\subpageref{NWsgp9-run9-1}}}\moddef{run blast~{\nwtagstyle{}\subpageref{NWsgp9-run9-1}}}\endmoddef\nwstartdeflinemarkup\nwusesondefline{\\{NWsgp9-sgp7-1}}\nwenddeflinemarkup
# Runnig blast
sub Run_Blast() \{
\} # END_SUB: Run_Blast
\nwindexdefn{\nwixident{{\&}Run{\_}Blast}}{:amRun:unBlast}{NWsgp9-run9-1}\eatline
\nwused{\\{NWsgp9-sgp7-1}}\nwidentdefs{\\{{\nwixident{{\&}Run{\_}Blast}}{:amRun:unBlast}}}\nwendcode{}\nwbegindocs{30}\nwdocspar
\sctn{Processing HSPs from {\Tt{}blast}}

\nwenddocs{}\nwbegincode{31}\sublabel{NWsgp9-extC-1}\nwmargintag{{\nwtagstyle{}\subpageref{NWsgp9-extC-1}}}\moddef{extract HSPs~{\nwtagstyle{}\subpageref{NWsgp9-extC-1}}}\endmoddef\nwstartdeflinemarkup\nwusesondefline{\\{NWsgp9-sgp7-1}}\nwenddeflinemarkup
# Extracting HSPs
sub Extract_HSP() \{
\} # END_SUB: Extract_HSP
\nwindexdefn{\nwixident{{\&}Extract{\_}HSP}}{:amExtract:unHSP}{NWsgp9-extC-1}\eatline
\nwused{\\{NWsgp9-sgp7-1}}\nwidentdefs{\\{{\nwixident{{\&}Extract{\_}HSP}}{:amExtract:unHSP}}}\nwendcode{}\nwbegindocs{32}\nwdocspar
\sctn{Running {\Tt{}geneid} on SRs}

\nwenddocs{}\nwbegincode{33}\sublabel{NWsgp9-runA-1}\nwmargintag{{\nwtagstyle{}\subpageref{NWsgp9-runA-1}}}\moddef{run geneid~{\nwtagstyle{}\subpageref{NWsgp9-runA-1}}}\endmoddef\nwstartdeflinemarkup\nwusesondefline{\\{NWsgp9-sgp7-1}}\nwenddeflinemarkup
# Running geneid
sub Run_geneid() \{
\} # END_SUB: Run_geneid
\nwindexdefn{\nwixident{{\&}Run{\_}geneid}}{:amRun:ungeneid}{NWsgp9-runA-1}\eatline
\nwused{\\{NWsgp9-sgp7-1}}\nwidentdefs{\\{{\nwixident{{\&}Run{\_}geneid}}{:amRun:ungeneid}}}\nwendcode{}\nwbegindocs{34}\nwdocspar
\sctn{Obtaining figures}

\nwenddocs{}\nwbegincode{35}\sublabel{NWsgp9-graG-1}\nwmargintag{{\nwtagstyle{}\subpageref{NWsgp9-graG-1}}}\moddef{graphical output~{\nwtagstyle{}\subpageref{NWsgp9-graG-1}}}\endmoddef\nwstartdeflinemarkup\nwusesondefline{\\{NWsgp9-sgp7-1}}\nwenddeflinemarkup
# Running gff2ps and aplot
sub Make_Plots() \{
\} # END_SUB: Make_Plots
\nwindexdefn{\nwixident{{\&}Make{\_}Plots}}{:amMake:unPlots}{NWsgp9-graG-1}\eatline
\nwused{\\{NWsgp9-sgp7-1}}\nwidentdefs{\\{{\nwixident{{\&}Make{\_}Plots}}{:amMake:unPlots}}}\nwendcode{}\nwbegindocs{36}\nwdocspar
%%%%%%%%%%%%%%%%%%%%%%%%%%%%%%%%%%%%%%%%APPENDIX

\newpage
\appendix

\sctn{Literate Programming}\vspace{2ex}

\begin{small}
\begin{quote}\slshape
I believe that the time is ripe for significantly better documentation
of programs, and that we can best achieve this by considering programs
to be works of literature. Hence, my title: "Literate Programming."
Let us change our traditional attitude to the construction of
programs: Instead of imagining that our main task is to instruct a
*computer* what to do, let us concentrate rather on explaining to
*human beings* what we want a computer to do.
\upshape\flushright --- \textbf{Donald Knuth}
\end{quote}\vspace{2ex}

\begin{quote}\slshape
Without wanting to be elitist, the thing that will prevent Literate
Programming from becoming a mainstream method is that it requires
thought and discipline. The mainstream is established by people who
want fast results while using roughly the same methods that everyone
else seems to be using, and Literate Programming is never going to
have that kind of appeal. This doesn't take away from its usefulness
as an approach.
\upshape\flushright --- \textbf{Patrick TJ McPhee}
\end{quote}\vspace{2ex}

\begin{quote}\slshape
A basic principle of data processing teaches the folly of trying to
maintain independent files in synchronism. It is far better to combine
them into one file with each record containing all the information
both files held concerning a given key.  Yet our practice in
programming documentation violates our own teaching. We typically
attempt to maintain a machine-readable form of a program and an
independent set of human-readable documents, consisting of prose and
flow charts. The results in fact confirm our teachings about the folly
of separate files.  Program documentation is notoriously poor, and its
maintenance is worse.  Changes made in the program do not promptly,
accurately, and invariably appear in the paper. The solution, I think,
is to merge the files, to incorporate the documentation in the source
program. This is at once a powerful incentive toward proper
maintenance, and an insurance that the documentation will always be
handy to the program user. Such programs are called
self-documenting...  
\upshape\flushright --- \textbf{Frederick Brooks}
\end{quote}\vspace{2ex}

\begin{quote}\slshape
Top-down design and successive refinement attack a programming task by
specifying it in the most general terms, then expanding these into
more and more specific and detailed actions, until the whole program
is complete. Structured design is the process of controlling the
overall design of a system or program so the pieces fit together
neatly, yet remain sufficiently decoupled that they may be
independently modified.  ... Each of these disciplines can materially
improve programmer productivity and the quality of code produced.
\upshape\flushright --- \textbf{Kernighan and Plauger}
\end{quote}\vspace{4ex}
\end{small}

We shall call our program \sgp{}, we are developing a parallel branch
for Sintenic Gene-Prediction, the other branch is developed by Thomas
Wiehe. Our new version is written as a {\Tt{}noweb} file named
{\Tt{}sgp2.nw}. From that file we can obtain both the code and the
documentation. A short script named {\Tt{}aol} converts the '{\Tt{}.nw}' to
'{\Tt{}.tex}' ---the \LaTeX{} documentation--- and '{\Tt{}.pl}' ---the real
program--- (Wow!!!). What {\Tt{}aol} does is:

\begin{enumerate}
\item Extracts program code from the {\Tt{}noweb} file:\\[1ex]
{\Tt{}notangle\ -L\ -Rsgp2.pl\ sgp2.nw\ >\ sgp2.tmp}
\item Swaps first line ({\Tt{}{\#}line}) with second ({\Tt{}{\#}!/}\ldots) to make it runnable:\\[1ex]
{\Tt{}perl\ -ne\ '{\$}.\ >\ 2\ \ {\&}{\&}\ do\ {\nwlbrace}\ print;\ next\ {\nwrbrace};\ \nwnewline
\ \ \ \ \ \ \ \ \ \ {\$}.\ ==\ 1\ {\&}{\&}\ do\ {\nwlbrace}\ split\ /\ /,\ {\$}{\_},\ 3;\ {\$}n\ =\ "{\$}{\_}[0]\ ".(++{\$}{\_}[1])."\ {\$}{\_}[2]"{\nwrbrace};\nwnewline
\ \ \ \ \ \ \ \ \ \ {\$}.\ ==\ 2\ {\&}{\&}\ (\ print\ "{\$}{\_}{\$}n"\ );\nwnewline
\ \ \ \ \ \ \ \ \ '\ sgp2.tmp\ >\ sgp2.pl}
\item Extracts program documentation and \LaTeX{}'ing it:\\[1ex]
{\Tt{}noweave\ -x\ sgp2.nw\ >\ sgp2.tex}\\[1ex]
{\Tt{}latex\ sgp2.tex\ ;\ dvips\ sgp2.dvi\ -o\ sgp2.ps}
\end{enumerate}

\newpage

\sctn{Man Pages}

The documentation for {\Tt{}perl} programs is written in a simple markup language called POD (Plain Old Documentation). Translators exist for {\Tt{}pod2man}, {\Tt{}pod2text}, {\Tt{}pod2html}, {\Tt{}pod2latex}, and {\Tt{}pod2fm}. 

The idea is to have the online documentation included with the source. That is possible with {\Tt{}{\_}{\_}DATA{\_}{\_}} tag, which defines a chunk of data within the same file.
%% TODO: define POD tags and which style was used for building up this man. 

\nwenddocs{}\nwbegincode{37}\sublabel{NWsgp9-PODK-1}\nwmargintag{{\nwtagstyle{}\subpageref{NWsgp9-PODK-1}}}\moddef{POD man page trailer~{\nwtagstyle{}\subpageref{NWsgp9-PODK-1}}}\endmoddef\nwstartdeflinemarkup\nwusesondefline{\\{NWsgp9-sgp7-1}}\nwenddeflinemarkup
__DATA__
=head1 NAME

    
$PROGRAM ($VERSION) - Improving Gene Prediction with Sinteny.

=head1 SYNOPSIS

    
    $PROGRAM [-hv] [-o 'options'] [-g 'options'] \\
      [-P filename] [-p filename] [-k filename] \\
      [-c value] [-s value] -1 seqfile_1 -2 seqfile_2

=head1 DESCRIPTION

=head1 OPTIONS

    

=over 4

=item B<-1> I<seqfile_1>

input file for first species.

=item B<-2> I<seqfile_2>

input file for second species.

=item B<-g>

geneid options

=item B<-o>

tblastx options

=item B<-c> I<value>

tblastx score cuttof

=item B<-s> I<value>

shrink hsp\\'s by value

=item B<-t> I<filename>

read tblastx file

=item B<-f> I<prefix>

read hsp gff files with in directory prefix and extension .hsp-rs

=item B<-k> I<prefix>

keep intermediate files with prefix

=item B<-p> I<filename>

ps output in filename file 

=item B<-P> I<filename>

geneid parameter file

=item B<-v>

verbose mode

=item B<-h>

produces this message

=back

=head1 FILES

=head1 DIAGNOSTICS

=head1 REQUIRES

=head1 BUGS

    
Report any problem to: B<jabril@imim.es>

=head1 AUTHOR

    
Roderic Guigo   : B<rguigo@imim.es>

Josep F. Abril  : B<jabril@imim.es>

B<$PROGRAM> is under GNU-GPL (C) 2000
\nwused{\\{NWsgp9-sgp7-1}}\nwidentuses{\\{{\nwixident{{\$}PROGRAM}}{:doPROGRAM}}\\{{\nwixident{{\$}VERSION}}{:doVERSION}}}\nwindexuse{\nwixident{{\$}PROGRAM}}{:doPROGRAM}{NWsgp9-PODK-1}\nwindexuse{\nwixident{{\$}VERSION}}{:doVERSION}{NWsgp9-PODK-1}\nwendcode{}

\nwixlogsorted{c}{{declare uses}{NWsgp9-decC-1}{\nwixu{NWsgp9-sgp7-1}\nwixd{NWsgp9-decC-1}\nwixd{NWsgp9-decC-2}\nwixd{NWsgp9-decC-3}\nwixd{NWsgp9-decC-4}}}%
\nwixlogsorted{c}{{declare variables}{NWsgp9-decH-1}{\nwixu{NWsgp9-sgp7-1}\nwixd{NWsgp9-decH-1}\nwixd{NWsgp9-decH-2}}}%
\nwixlogsorted{c}{{define defaults}{NWsgp9-defF-1}{\nwixu{NWsgp9-sgp7-1}\nwixd{NWsgp9-defF-1}}}%
\nwixlogsorted{c}{{extract HSPs}{NWsgp9-extC-1}{\nwixu{NWsgp9-sgp7-1}\nwixd{NWsgp9-extC-1}}}%
\nwixlogsorted{c}{{get options}{NWsgp9-getB-1}{\nwixu{NWsgp9-sgp7-1}\nwixd{NWsgp9-getB-1}}}%
\nwixlogsorted{c}{{global subroutines}{NWsgp9-gloI-1}{\nwixu{NWsgp9-sgp7-1}\nwixd{NWsgp9-gloI-1}\nwixd{NWsgp9-gloI-2}\nwixd{NWsgp9-gloI-3}}}%
\nwixlogsorted{c}{{graphical output}{NWsgp9-graG-1}{\nwixu{NWsgp9-sgp7-1}\nwixd{NWsgp9-graG-1}}}%
\nwixlogsorted{c}{{Main Loop}{NWsgp9-Mai9-1}{\nwixu{NWsgp9-sgp7-1}\nwixd{NWsgp9-Mai9-1}}}%
\nwixlogsorted{c}{{POD man page trailer}{NWsgp9-PODK-1}{\nwixu{NWsgp9-sgp7-1}\nwixd{NWsgp9-PODK-1}}}%
\nwixlogsorted{c}{{run blast}{NWsgp9-run9-1}{\nwixu{NWsgp9-sgp7-1}\nwixd{NWsgp9-run9-1}}}%
\nwixlogsorted{c}{{run geneid}{NWsgp9-runA-1}{\nwixu{NWsgp9-sgp7-1}\nwixd{NWsgp9-runA-1}}}%
\nwixlogsorted{c}{{sgp2.pl}{NWsgp9-sgp7-1}{\nwixd{NWsgp9-sgp7-1}}}%
\nwixlogsorted{c}{{shebang}{NWsgp9-she7-1}{\nwixu{NWsgp9-sgp7-1}\nwixd{NWsgp9-she7-1}}}%
\nwixlogsorted{c}{{version control}{NWsgp9-verF-1}{\nwixu{NWsgp9-sgp7-1}\nwixd{NWsgp9-verF-1}}}%
\nwixlogsorted{i}{{\nwixident{{\$}blast{\_}opt}}{:doblast:unopt}}%
\nwixlogsorted{i}{{\nwixident{{\&}Extract{\_}HSP}}{:amExtract:unHSP}}%
\nwixlogsorted{i}{{\nwixident{{\&}fill{\_}left}}{:amfill:unleft}}%
\nwixlogsorted{i}{{\nwixident{{\&}fill{\_}right}}{:amfill:unright}}%
\nwixlogsorted{i}{{\nwixident{{\$}geneid{\_}opt}}{:dogeneid:unopt}}%
\nwixlogsorted{i}{{\nwixident{{\$}geneid{\_}param}}{:dogeneid:unparam}}%
\nwixlogsorted{i}{{\nwixident{{\&}get{\_}exec{\_}time}}{:amget:unexec:untime}}%
\nwixlogsorted{i}{{\nwixident{{\$}help{\_}flg}}{:dohelp:unflg}}%
\nwixlogsorted{i}{{\nwixident{{\$}hsp}}{:dohsp}}%
\nwixlogsorted{i}{{\nwixident{{\&}Make{\_}Plots}}{:amMake:unPlots}}%
\nwixlogsorted{i}{{\nwixident{{\&}max}}{:ammax}}%
\nwixlogsorted{i}{{\nwixident{{\$}NF}}{:doNF}}%
\nwixlogsorted{i}{{\nwixident{{\$}ofn}}{:doofn}}%
\nwixlogsorted{i}{{\nwixident{{\$}PROGRAM}}{:doPROGRAM}}%
\nwixlogsorted{i}{{\nwixident{{\&}prt{\_}foeprg}}{:amprt:unfoeprg}}%
\nwixlogsorted{i}{{\nwixident{{\&}prt{\_}Help}}{:amprt:unHelp}}%
\nwixlogsorted{i}{{\nwixident{{\&}prt{\_}progress}}{:amprt:unprogress}}%
\nwixlogsorted{i}{{\nwixident{{\$}ps{\_}output}}{:dops:unoutput}}%
\nwixlogsorted{i}{{\nwixident{{\&}Run{\_}Blast}}{:amRun:unBlast}}%
\nwixlogsorted{i}{{\nwixident{{\&}Run{\_}geneid}}{:amRun:ungeneid}}%
\nwixlogsorted{i}{{\nwixident{{\$}score{\_}cutoff}}{:doscore:uncutoff}}%
\nwixlogsorted{i}{{\nwixident{{\$}seq1}}{:doseq1}}%
\nwixlogsorted{i}{{\nwixident{{\$}seq2}}{:doseq2}}%
\nwixlogsorted{i}{{\nwixident{{\$}SGP2}}{:doSGP2}}%
\nwixlogsorted{i}{{\nwixident{{\$}shrink}}{:doshrink}}%
\nwixlogsorted{i}{{\nwixident{{\$}tbx}}{:dotbx}}%
\nwixlogsorted{i}{{\nwixident{@tmp{\_}ver}}{@tmp:unver}}%
\nwixlogsorted{i}{{\nwixident{{\$}verbose{\_}flg}}{:doverbose:unflg}}%
\nwixlogsorted{i}{{\nwixident{{\$}VERSION}}{:doVERSION}}%
\nwixlogsorted{i}{{\nwixident{{\&}Which{\_}Options}}{:amWhich:unOptions}}%
\nwbegindocs{38}%$

\end{document}


\nwenddocs{}
