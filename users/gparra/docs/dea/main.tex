\documentclass[11pt]{article}
%
%2345678901234567890123456789012345678901234567890123456789012345678901234567890
%        1         2         3         4         5         6         7         8
%
% $Id: main.tex,v 1.2 2003-09-22 17:13:27 gparra Exp $
%

\usepackage[a4paper,offset={0pt,0pt},hmargin={3.5cm,3cm},vmargin={2.5cm,2.5cm}]{geometry}
\usepackage{graphics}
\usepackage[dvips]{graphicx}
%% pstricks
\usepackage[dvips]{pstcol}
\usepackage{pstricks}
%% bibliography
\usepackage{natbib}
%% latex2html
\usepackage{url}
\usepackage{html}     
\usepackage{htmllist} 
%% tables    
%\usepackage{colortbl}
%\usepackage{multirow}
%\usepackage{hhline}
%\usepackage{tabularx}
%\usepackage{dcolumn}

%% layout
\usepackage{fancyhdr} % Do not use \usepackage{fancybox} -> TOCs disappear
%\usepackage{lscape}
\usepackage{rotating}
%\usepackage{multicol}
%% fonts
%\usepackage{times}\fontfamily{ptm}\selectfont
%\usepackage{t1enc}

% New Commands are defined here
\newcommand{\sctn}[1]{\section{#1}}
\newcommand{\subsctn}[1]{\subsection{#1}}
\newcommand{\subsubsctn}[1]{\subsubsection{#1}}
\newcommand{\desc}[1]{\item[#1] \ \\}

% PSTRICKs definitions
\pslongbox{ExFrame}{\psframebox}
\newcommand{\cln}[1]{\fcolorbox{black}{#1}{\textcolor{#1}{\rule[-.3ex]{1cm}{1ex}}}}
\newpsobject{showgrid}{psgrid}{subgriddiv=0,griddots=1,gridlabels=6pt}

%%%%% global urls
% \newcommand{\getpsf}[1]{\html{(\htmladdnormallink{Get PostScript file}{./Psfiles/#1})}}   
\def\mtgparra{\htmladdnormallink{\textbf{gparra@imim.es}}{MAILTO:gparra@imim.es}}

% defs
\def\dicty{\textit{Dictyostelium discoideum}}
\def\dic{\textit{D. discoideum}}
\def\fugu{\textit{Fugu rubripes}}
\def\fug{\textit{F. rubripes}}
\def\drome{\textit{Drosophila melanogaster}}
\def\dme{\textit{D. melanogaster}}
\def\dro{\textit{Drosophila}}
\def\tetra{\textit{Tetraodon nigroviridis}}
\def\danio{\textit{Brachydanio rerio}}
\def\homo{\textit{Homo sapiens}}
\def\arab{\textit{Arabidopsis thaliana}}
\def\wheat{\textit{Triticum aestivum}}
\def\bn{\textsc{blastn}}
\def\tbx{\textsc{tblastx}}
\def\ps{\textsc{PostScript}}
\begin{latexonly}
\def\geneid{\texttt{geneid}}
\end{latexonly}
\begin{htmlonly}
\def\geneid{\bold{geneid}}
\end{htmlonly}

\def\genscan{\texttt{genscan}}
\def\sgp{\texttt{SGP2}}
\def\genos{\textbf{Genoscope}}

% Setting text for footers and headers

\def\tit{\textsc{Computational identification of genes}}
\fancyhead{} % clear all fields
\fancyfoot{} % clear all fields
\fancyhead[RO,LE]{\thepage}
\fancyhead[LO,RE]{\tit}
%\fancyfoot[LO,LE]{\small\textsl{Gen\'{\i}s Parra}}
%\fancyfoot[RO,RE]{\small\textbf{\today}}
\renewcommand{\headrulewidth}{1pt}
%\renewcommand{\footrulewidth}{1pt}

%%%%%%%%%%%%%%%%%%%%%%%%%%%%%%%%%%%%%%%%%%%%%%%%%%%%%%%%%%%%%%%%%%%%%%%%%%%
\input Colores

\begin{document}
\bodytext{text="#000000" bgcolor="#FFFF99"}

\thispagestyle{empty}

\begin{titlepage}

\begin{center}
\vfill
\begin{latexonly}
\psframebox[framearc=0.5,fillstyle=solid,fillcolor=lightorange]{
\parbox[b][2.75cm][c]{7cm}{
{\textbf{\Large \shortstack[c]{Research project \\[1ex]for the \\[1ex]Advanced Studies Diploma}}}}}\\[12ex]
\end{latexonly}

\begin{rawhtml}
<BR>
\end{rawhtml}

\textbf{\huge Computational identification of genes}\\[12ex]


%\textbf{\Large Authors List Here}\\[1ex]
\textbf{\Large Gen\'{\i}s Parra Farr\'e\raisebox{0.85ex}{\footnotesize$\dag\,$}}\\[2ex]
\textbf{Ph.D. Advisor: Roderic Guig\'o Serra}\\[5ex]

 % \raisebox{0.85ex}{\footnotesize$\,\dag$}\\[0.5ex]

\textbf{\large --- March 10, 2003 ---}\\[10ex]


\begin{abstract}
\begin{center}
\parbox{0.75\linewidth}{

Development of a new version of the program \geneid\ which identifies
complete exon/intron structures of genes in genomic DNA is
presented. The parameter file describing fundamental translational and
splicing signals as well as coding potential properties used to
predict genes is described. Distinct sets of parameters are derived
for \drome\ and \dicty. Applications of the program to find genes in
genomic regions are discussed and analysed.

} % parbox
\end{center}
\end{abstract}
 
\vfill


\vfill

\begin{raggedleft}
\scalebox{0.9 1}{\Large\textsl{\textbf{Genome Informatics Research Lab}}}\\
Grup de Recerca en Infom\`atica Biom\`edica\\
Institut Municipal d'Investigaci\'o M\`edica\\
Centre de Regulaci\'o Gen\`omica \\
Universitat Pompeu Fabra\\[2ex]
\raisebox{0.85ex}{\footnotesize$\dag\,$}{\large e-mail: \mtgparra}\\
\end{raggedleft}
\end{center}

\end{titlepage} %'

%%%%%%%%%%%%%%%%%%%% FRONTMATTER

\newpage
\pagenumbering{roman}
\setcounter{page}{1}
\pagestyle{fancy}
% Marks redefinition must go here because pagestyle 
% resets the values to the default ones.
\renewcommand{\sectionmark}[1]{\markboth{}{\thesection.\ #1}}
\renewcommand{\subsectionmark}[1]{\markboth{}{\thesubsection.\ \textsl{#1}}}

\tableofcontents
\begin{itemize}
\item \geneid\ {\it in Drosophila.}
\item Sequence and analysis of chromosome 2 of \dicty.
\end{itemize}

\begin{latexonly}
\vfill
\begin{center}
{\tiny$<$ \verb$Id: main.tex,v 1.2 2003-09-22 17:13:27 gparra Exp $$>$ }
\end{center}
\end{latexonly}

%%%%%%%%%%%%%%%%%%%% MAINMATTER

\newpage
\pagenumbering{arabic}
\setcounter{page}{1}
\ \\
\sctn{Introduction}
With the complete sequences of many genomes in our hands, the
challenge ahead is to extract relevant information encoded within the
billions of nucleotides stored in our databases.

Computational methods for automated genome annotation are critical to
our community\'{ }s ability to make full use of the large volume of
genomic sequence being generated and released. Raw genomic sequences
are mainly useless for the scientific community. The goal of the
annotation process is to assign as much information as possible to the
raw sequence of complete genomes.

In a very simplistic way, the first step would be to find the
collection of genes encoded in the DNA sequence. The next step would be
to assign a function to each possible protein, where the three dimensional
structure of the proteins will play a key role. And finally, establish
the network of interactions and regulations among all the proteins of
a complete genome.

The following work is focused on the first step: to find where genes are.
This can be accomplished by ab initio gene finding, by identify
homologies to known genes from other organism, by the alignment of
full-length or partial mRNA sequences to the genomic DNA, or through
combinations of such methods.

A large body of literature on the subject of gene prediction has been
accumulated in the last fifteen years (very interesting reviews
in \cite{burge:1998a} and \cite{stormo:2000a}). Nowadays, there is a
large number of gene finding programs raising the obvious question of
whether the gene finding problem has perhaps already been solved by
one or more of these programs. This question was definitively answered
in the negatively in a recent number of papers
\citep{guigo:2000c,rogic:2001a, guigo:2003a}.

The main focus research interest of our group is  the investigation
of the signals involved in gene specification in genomic sequences
(promoter elements, splice sites, translation initiation sites,
...). We are interested both in the mechanism of their recognition and
processing, and in their evolution. In addition, but related to this
basic component of our research, our group is also involved in the
development of software for gene prediction and annotation in genomic
sequences.

\geneid\ \citep{guigo:1992a} was one of the first programs to predict
full exonic structures of vertebrates genes in anonymous DNA
sequences. Since the first version there had been substantial
developments in the field of computational gene
identification. Therefore, when I started my Ph.D. at Roderic's group,
the first task to do in collaboration with Enrique Blanco was to
develop an improved version of \geneid. Enrique was in charge of the
algorithmic and programing part while I was in charge of building a
model for the biological recognition of the signals and the coding
statistic bias. Half a year later \geneid\ version 1.0 was released
showing an accuracy similar to the other existing tools but much more
efficient at handing very large genomic sequences.


\newpage
\ \\
\sctn{Published Papers}

Both papers presented here are mainly based on \geneid, the
analysis its accuracy and the set of predicted genes.

In the first paper, it is explained the ``training'' of \geneid\ on
\drome\ and the accuracy of the predicted genic structures in a 2.9 Mb
genomic region. The second one contains the application of \geneid\
in the annotation of the chromosome 2 of \dicty.

\subsctn{\geneid\ in Drosophila}

\begin{latexonly}
\psframebox[framearc=0.5,fillstyle=solid,fillcolor=lightorange]{
\parbox[b][1.7cm][c]{6.75cm}{
   \shortstack[l]{
        G. Parra, E. Blanco, and R. Guig\'o. \\
        {\geneid\ \it in Drosophila.} \\
        Genome Research 10(4):511-515 (2000)}}}
\\
\end{latexonly}

\begin{htmlonly}
\parbox[b][1.7cm][c]{6.75cm}{
   \shortstack[l]{
        G. Parra, E. Blanco, and R. Guig\'o. \\
        {\geneid\ \it in Drosophila.} \\
        Genome Research 10(4):511-515 (2000)}}
\\
\end{htmlonly}
\begin{rawhtml}
<BR>
\end{rawhtml}


This paper must be considered in the context of the Genome Assessment
Project (GASP,
\htmladdnormallink{http://www.fruitfly.org/GASP1/}{http://www.fruitfly.org/GASP1/}
). The GASP experiment was organized to formulate guidelines and
accuracy standards for evaluating computational tools and to encourage
the development of existing approaches through a careful assessment
and comparison of the predictions made by programs at that time.

The GASP experiment consisted on the following stages:
\begin{itemize}
\item Training data for the {\it Adh} region, including 2.9 Mb of
\drome\ genomic sequence was collected by the organizers and provided
to the participants.
\item A set of annotations based on experimental data was
developed to evaluate submissions while the participating groups
produced and submitted their annotations for the region.
\item The participant group\'{ }s predictions were compared to the
standards and the results were presented as a tutorial at the
Intelligent Systems for Molecular Biology (ISMB, Heidelberg 1999).
\end{itemize}

Additional papers in a special issue of the { \it Genome Research}
magazine were written by the participants and describe their methods
and the results in detail. Our paper was included in this special
issue and explains how the parameters needed for \geneid\ to predict
genes in \drome\ were computed. The final \geneid\ predictions showed
an accuracy comparable to the gene finding programs that exhibited the
highest accuracy in the GASP results published in \cite{reese:2000a}.

Although \geneid\ was not used by the Drosophila Genome Project to
annotate the \drome\ genome, it had some usage through our web page
and from people who had freely download the program. Some experimental
papers have based their work on \geneid\ predictions
\citep{dunlop:2000a,castellano:2001a,beltran:2003a}.



\subsctn{Sequence and analysis of chromosome 2 of \dicty}
\begin{latexonly}
\psframebox[framearc=0.5,fillstyle=solid,fillcolor=lightorange]{
\parbox[b][3.2cm][c]{14.5cm}{
        \shortstack[l]{
        G. Gl\"okner, L. Eichinger, K. Szafranski, J.A. Pachebat,
        A.T. Bankier, P.H. Dear,\\ 
         R. Lehmann, C. Baumgart, G. Parra, J.F. Abril, R. Guig\'o, K. Kumpf, 
        B. Tunggal, \\
        E. Cox, M.A. Quail, the Dictyostelium Genome Sequencing Consortium,  
        \\ M. Platzer, A. Rosenthal and A.A. Noegel.\\
        {\it Sequence and Analysis of Chromosome 2 of Dictyostelium
        discoideum.}\\
        Nature 418(6893):79-85 (2002)}}}
\\
\end{latexonly}

\begin{htmlonly}
\parbox[b][3.2cm][c]{14.5cm}{
        \shortstack[l]{
        G. Gl\"okner, L. Eichinger, K. Szafranski, J.A. Pachebat,
        A.T. Bankier, P.H. Dear,\\ 
         R. Lehmann, C. Baumgart, G. Parra, J.F. Abril, R. Guig\'o, K. Kumpf, 
        B. Tunggal, \\
        E. Cox, M.A. Quail, the Dictyostelium Genome Sequencing Consortium,  
        \\ M. Platzer, A. Rosenthal and A.A. Noegel.\\
        {\it Sequence and Analysis of Chromosome 2 of Dictyostelium
        discoideum.}\\
        Nature 418(6893):79-85 (2002)}}}
\\
\end{htmlonly}
\begin{rawhtml}
<BR>
\end{rawhtml}


This paper was done in collaboration with the Dictyostelium Genome
Sequencing Consortium, which is an international collaboration between
the University of Cologne, the Institute of Molecular Biotechnology in
Jena, the Baylor College of Medicine in Houston, Institut Pasteur in
Paris, and the Sanger Centre in Hinxton for the sequencing and the
analysis of the genome of \dicty. 

\dicty\ is a soil-living amoeba. The hereditary information is carried
on six chromosomes with sizes ranging from 4 to 7 Mb resulting in a
total of about 34 Mb of haploid DNA genome with a base composition of
77\% [A+T]. This extreme base compositon biased to A+T nucleotides
gave us the opportunity to discover how a high A+T content can
influence the signals and the codon usage which are the landmarks for
gene prediction.

Obviously with such a biased base composition the performance of
current gene prediction programs were rather poor. Our group developed
a parameter file for \geneid\ based on experimental annotated
sequences from \dicty.

In this paper it is not explained the training of \geneid\ that
basically followed the same protocol that the training on \drome\
sequences, but there is a complete analysis of the \geneid\ predicted
proteins. The annotation of the 2,799 genes of the chromosome 2 of
\dicty\ was based on \geneid. The paper is focused on the analysis of
the function and the structure of \geneid\ predicted genes. This
analysis reinforce the view that the evolutionary position of \dicty\
is located before the branching of metazoa and fungi but before the
divergence of the plant kingdom.




\sctn{Future Work}

\subsctn{Gene definition across different species}

Each genome seems to have its proprietary signatures for gene
recognition which was and is shaped by environmental pressures during
evolution. This implies that gene structures and the transcription and
translation machinery in each species are adapted to each other
enabling the cell appropriate transcription and translation for each
genome. Our plans include to train \geneid\ to predict genes on
different species.  Since ``geneid in Drosophila'', we have developed
parameter files for different species: human, \fugu, \wheat\ and \tetra.
The idea is to obtain parameters for other model organism an try to
establish some evolutionary relations among the way genes are defined
in different evolutionary groups.

\subsctn{Syntenic gene prediction} 

The increasing number of available genomes has lead to the development
of new computational gene finding methods that use sequence
conservation to improve the accuracy of gene prediction methods
\citep{miller:2001a,rinner:2002a}.  Anonymous genomic sequences are
compared against anonymous genomic sequences from different organism,
under the assumption that regions conserved in the sequence will tend
to correspond to coding exons from homologous genes.

We are working on the development of a new gene finding program called
Syntenic Gene Prediction 2 (\sgp). \sgp\ is a tool that combines ab
initio gene prediction, coming from \geneid, with comparison of
genomic sequences, using \tbx\ program (WU-BLAST,
\htmladdnormallink{http://blast.wustl.edu/}{http://blast.wustl.edu/}).
Our preliminary results suggest that this method outperforms purely ab
initio gene finding methods. \sgp\ was used to generate a complete set
of gene predictions on both human and mouse by comparing the genomes
of these two species.

\bibliographystyle{apalike}
\bibliography{main}

\newpage
\appendix
 
\sctn{Appendix - Papers}
\begin{htmlonly}
\htmladdnormallink{{\geneid\ \it in Drosophila.\\}}{./parra_GR_2000.pdf} 

G. Parra, E. Blanco, and R. Guig\'o. \\
Genome Research 10(4):511-515 (2000)


\htmladdnormallink{{\it Sequence and Analysis of Chromosome 2 of
        Dictyostelium discoideum.\\}}{./gernot_Nature_2002.pdf} 

G. Gl\"okner, L. Eichinger, K. Szafranski, J.A. Pachebat,
A.T. Bankier, P.H. Dear, R. Lehmann, C. Baumgart, G. Parra,
J.F. Abril, R. Guig\'o, K. Kumpf, B. Tunggal,E. Cox, M.A. Quail, the
Dictyostelium Genome Sequencing Consortium,  M. Platzer,
A. Rosenthal and A.A. Noegel.\\
Nature 418(6893):79-85 (2002)

\end{htmlonly}


\end{document}













