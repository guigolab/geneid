\documentclass[11pt]{article}
%
%2345678901234567890123456789012345678901234567890123456789012345678901234567890
%        1         2         3         4         5         6         7         8
%
% $Id: GFF2APLOT_MANUAL.tex,v 1.4 2001-11-06 17:38:27 jabril Exp $
%
% latex GFF2APLOT_MANUAL.tex ; dvips GFF2APLOT_MANUAL.dvi -o GFF2APLOT_MANUAL.ps -t a4 ;
%
\usepackage{noweb}
\usepackage[a4paper,offset={0pt,0pt},hmargin={2.5cm,2cm},vmargin={2cm,2cm}]{geometry}
\usepackage{graphics}
\usepackage[dvips]{graphicx}
%% pstricks
\usepackage[dvips]{pstcol}
\usepackage{pstricks}
%\usepackage{pst-node}
%\usepackage{pst-char}
%\usepackage{pst-grad}
%% bibliography
\usepackage{natbib}
%% latex2html
\usepackage{url}
\usepackage{html}     
\usepackage{htmllist} 
%% tables    
%\usepackage{colortbl}
\usepackage{multirow}
%\usepackage{hhline}
%\usepackage{tabularx}
\usepackage{dcolumn}
%% seminar
%\usepackage{semcolor,semlayer,semrot,semhelv,sem-page,slidesec}
%% draft watermark
%\usepackage[all,dvips]{draftcopy}
%\draftcopySetGrey{0.9}
%\draftcopyName{CONFIDENTIAL}{100}
%% layout
%\usepackage{fancyheadings}
%\usepackage{fancybox}
\usepackage{fancyhdr} % Do not use \usepackage{fancybox} -> TOCs disappear
\usepackage{lscape}
%\usepackage{rotating}
%\usepackage{multicol}
\usepackage{verbatim}
%\usepackage{version}
%% fonts
\usepackage{times}\fontfamily{ptm}\selectfont
\usepackage{t1enc}
 
% Colors for gff2ps
\input tables/AplotColorDefs.tex

% New Commands are defined here
\input ./tabledefs.tex

\newcommand{\sctn}[1]{\section{#1}}
\newcommand{\subsctn}[1]{\subsection{#1}}
\newcommand{\subsubsctn}[1]{\subsubsection{#1}}
\newcommand{\desc}[1]{\item[#1] \ \\}

\newcommand{\pa}[1]{{\footnotesize\textsf{$<$\textsl{#1}$>$}}}
% \newcommand{\op}[1]{{\bfseries\sffamily #1}}
\newcommand{\tp}[1]{{'{\bfseries\sffamily #1}'}}
\newcommand{\pr}[1]{{\mdseries\sffamily\slshape $<$#1$>$}}

% PSTRICKs definitions
\pslongbox{ExFrame}{\psframebox}
\newcommand{\cln}[1]{\fcolorbox{black}{#1}{\textcolor{#1}{\rule[-.3ex]{1cm}{1ex}}}}
\newpsobject{showgrid}{psgrid}{subgriddiv=0,griddots=1,gridlabels=6pt}
% \pscharpath[fillstyle=solid, fillcolor=verydarkcyan, linecolor=black, linewidth=1pt]{\sffamily\scshape\bfseries\veryHuge #1 }

%%%%% global urls
% \newcommand{\getpsf}[1]{\html{(\htmladdnormallink{Get PostScript file}{./Psfiles/#1})}}   
\def\mtjabril{\htmladdnormallink{\textbf{jabril@imim.es}}{MAILTO:jabril@imim.es}}

% defs
\def\prog{\textsc{\textbf{gff2aplot}}}
\def\ps{\textsc{PostScript}}
\def\eoline{$\backslash\backslash$}

% Setting text for footers and headers
\fancyhead{} % clear all fields
\fancyfoot{} % clear all fields
\fancyhead[RO,LE]{\thepage}
\fancyhead[LO,RE]{\rightmark}
\fancyfoot[LO,LE]{\small\textsl{Abril, J.F.}}
\fancyfoot[RO,RE]{\small\textbf{\today}}
\renewcommand{\headrulewidth}{1pt}
\renewcommand{\footrulewidth}{1pt}

%%%%%%%%%%%%%%%%%%%%%%%%%%%%%%%%%%%%%%%%%%%%%%%%%%%%%%%%%%%%%%%%%%%%%%%%%%%

\begin{document}
\thispagestyle{empty}

\begin{titlepage}

\ \vfill
\begin{center}
\fbox{\parbox{0.6\linewidth}{\centering\textbf{\Huge \\[2ex]{\prog}\\[2ex]\textsc{User's Manual}\\[3ex]}}}\\[5ex]

\textbf{\Large Josep F. Abril}\raisebox{0.85ex}{\footnotesize$\,\dag$}\\[5ex]

\textbf{\large --- \today ---}\\[10ex]

\vskip 5ex

{\large$<$\verb$Id: GFF2APLOT_MANUAL.tex,v 1.4 2001-11-06 17:38:27 jabril Exp $$>$ }

\vfill

\begin{flushright}
\scalebox{0.9 1}{\Large\textsl{\textbf{Genome Informatics Research Lab}}}\\
Grup de Recerca en Infom\`atica Biom\`edica\\
Institut Municipal d'Investigaci\'o M\`edica\\
Universitat Pompeu Fabra\\[2ex]
\raisebox{0.85ex}{\footnotesize$\dag\,$}{\large e-mail: \mtjabril}\\
\end{flushright}
\end{center}

\end{titlepage}

%%%%%%%%%%%%%%%%%%%%% FRONTMATTER: License, Acknowledgements, TOC, LOF, and LOT.

\newpage
\pagenumbering{roman}
\setcounter{page}{1}
\pagestyle{fancy}
% Marks redefinition must go here because pagestyle 
% resets the values to the default ones.
\renewcommand{\sectionmark}[1]{\markboth{}{\textbf{\prog}\hspace{4ex}\thesection.\ #1}} 
\renewcommand{\subsectionmark}[1]{\markboth{}{\textbf{\prog}\hspace{4ex}\thesubsection.\ \textsl{#1}}}

% %% Copyright Advice
% \input GNU_GPL.tex

% \newpage

% %% Acknowledgements
% \sctm{Acknowledgements}
% \input Acknowledgements.tex 

\tableofcontents

\listoffigures

\listoftables

%%%%%%%%%%%%%%%%%%%% MAINMATTER: CHAPTERS 

\newpage
\pagenumbering{arabic}
\setcounter{page}{1}

%
% \include{Introduction.tex}

%
% %
% Basics.tex
%
% About GFF format and GFF2PS.
%
% $Id: Basics.tex,v 1.2 2003-03-04 18:55:17 jabril Exp $
%

\chap{Basic Concepts}

In this chapter we would introduce you to GFF format definition as
well as how {\prog} handles your GFF files. The main goal is that you
will learn which is the proper record structure for GFF files. So
that, you will understand the program warnings when a record does not
fit such structure and being not drawn. {\prog} rejects all those
records not conforming with the record structures described in the
following sections.

	\sctn{GFF philosophy}

	%	%
% Philosophy_GFF.tex
%
% Brief description on GFF-format.
%
% $Id: Philosophy_GFF.tex,v 1.1 2002-12-24 13:07:37 jabril Exp $
%
% \pslongbox{mypsframe}{\psframebox[framearc=0,linecolor=darkred,linewidth=2pt,fillstyle=solid,fillcolor=verylightgreen,framesep=2ex]}

\label{sec:philoGFF}
\newcommand{\IR}[1]{ %
  \rule{2.5em}{0pt}\makebox[0cm][c]{\rotatebox{35}{\ $<$#1$>$}}
  } % #1 string rotated

GFF Protocol Specification was initially proposed by Richard Durbin and David Haussler with amendments proposed by Lincoln Stein, Suzana Lewis, Anders Krogh and others. The GFF Specification is nowadays maintained at the Sanger Center by Richard Bruskievich\footnote{\texttt{rbsk@sanger.ac.uk}}.\par

We have developed \prog\ under \textbf{GFF Version 1} and \textbf{GFF Version 2} specifications, here we are going to summarize the basic aspects for it. You can obtain a more complete definition at Sanger Center GFF format page at:

  \centerline{\bfseries\texttt{http://www.sanger.ac.uk/Software/GFF/gff.shtml}} \vspace{0.5ex}

and the full specs at:

  \centerline{\bfseries\texttt{http://www.sanger.ac.uk/Software/GFF/GFF\_Spec.shtml}} \vspace{0.5ex}

There is a mailing list to which you can send comments, enquiries, complaints, etc\ldots about GFF. To be added to the mailing list you have to send a mail to \texttt{\bfseries Majordomo@sanger.ac.uk} with the following command in the body of your email message: \texttt{\bfseries subscribe gff-list}\par

\subsubsctn{GFF Format Definition:}\label{sec:GFFdefs}
\textbf{GFF} ---General Feature Format (formerly called 'Gene Feature Finding')--- is a format specification for describing genes and other features associated with genomic sequences and the transfer of feature information. A GFF record is an extension of a basic (name,start,end) tuple (or "NSE") that can be used to identify a substring of a biological sequence. (For example, the NSE (ChromosomeI,2000,3000) specifies the third kilobase of the sequence named "ChromosomeI"). Filename extension for GFF format has been defined as `.gff'.\par
GFF format is not conceived to be used for complete data management of the analysis and annotation of genomic sequences, there are other much powerful systems developed for that (as Acedb, Genotator, \ldots). This format is intended to be easy to parse and process by a variety of programs in different languages ---i.e. Unix tools like grep, sort and simple perl and awk scripts---. So, GFF format has a record-based structure, where each feature is described on a single line, and line order is not relevant. In Version 2, all the fields of every single record must be separated by TAB characters (`$\backslash$t'), revoking previous permissions to use arbitrary whitespaces as field delimiters. Table~\ref{GFFrecords} placed below, shows some simple example records.\par

\begin{table}[!ht]
\caption{The simplest GFF-format records.}
\label{GFFrecords}
\begin{center}
\begin{mypsframe}
\begin{minipage}[!ht][][c]{0.8\linewidth}
\scriptsize\ttfamily
\begin{tabbing}
12345678\=12345678\=12345678\=12345678\=12345678\=12345678\=12345678\=12345678\=12345678\=12345678\=\kill
SEQ1\>netgene\>splice5\>172\>173\>0.94\>+\>.\>\\
SEQ1\>genie\>sp5-20\>163\>182\>2.3\>+\>.\>\\
SEQ1\>genie\>sp5-10\>168\>177\>2.1\>+\>.\>\\
SEQ2\>grail\>ATG\>17\>19\>2.1\>-\>0\>\\
SEQ1\>EMBL\>atg\>103\>105\>.\>+\>0\>labl\>\#\\
SEQ1\>EMBL\>exon\>103\>172\>.\>+\>0\>labl\>\#\\
SEQ1\>EMBL\>splice5\>172\>173\>.\>+\>.\>.\>\#
\end{tabbing}
\end{minipage}
\end{mypsframe}
\end{center}
\end{table}

\subsubsctn{Fields Definition:}

Fields are:\\[1ex]
\centerline{\small\hspace{-0.25cm}\framebox{\hspace{0.5cm}\pa{seqname} \pa{source} \pa{feature} \pa{start} \pa{end} \pa{score} \pa{strand} \pa{frame} \symbol{91}\textmd{\textsf{\textsl{group}}}\symbol{93} \symbol{91}\ldots\symbol{93}\hspace{0.5cm}}}

\begin{description}
\item[$<$seqname$>$] The name of the sequence. Having an explicit sequence name allows a feature file to be prepared for a data set of multiple sequences. You can use the identifier of the sequence in an accompanying file containing the sequence nucleotides string, or the identifier for a sequence in a public database ---as EMBL/Genbank/DDBJ accession number---.
\item[$<$source$>$] The source of this feature. This field will normally be used to indicate the program making the prediction, or if it comes from public database annotation, or is experimentally verified, etc\ldots
\item[$<$feature$>$] The feature type name. As you can use other features, it would be desirable to have a Standard Table for common features. For this standard table has been proposed to fall back on the international public standards for genomic database feature annotation, specifically, the DDBJ/EMBL/GenBank feature table\footnote{See the DDBJ/EMBL/GenBank feature key table definition at:\\\centerline{\bfseries\texttt{http://www.ebi.ac.uk/embl/Documentation/FT\_definitions/feature\_table.html}}}. Some of the most used terms in genomics from that table are summarized at Appendix~\ref{sec:gff-featbl}.
\item[$<$start$>$, $<$end$>$] Integers. $<$start$>$ must be less than or equal to $<$end$>$, so reverse strand coordinates must be defined in forward coords. In GFF Version 1 sequence numbering starts at 1, so these numbers should be between 1 and the length of the relevant sequence, inclusive. Version 2 condones values of $<$start$>$ and $<$end$>$ that extend outside the reference sequence. 
\item[$<$score$>$] A floating point value. When there is no score you should use `.'.
\item[$<$strand$>$] One of `+', `-' or `.'. `.' should be used when strand is not relevant.
\item[$<$frame$>$] One of `0', `1', `2' or `.'. `0' indicates that the specified region is in frame, i.e. that its first base corresponds to the first base of a codon. `1' indicates that there is one extra base, i.e. that the second base of the region corresponds to the first base of a codon, and `2' means that the third base of the region is the first base of a codon. If the strand is `-', then the first base of the region is value of $<$end$>$, because the corresponding coding region will run from $<$end$>$ to $<$start$>$ on the reverse strand. As with $<$strand$>$, if the frame is not relevant then set $<$frame$>$ to `.'. It has been pointed out that "phase" might be a better descriptor than "frame" for this field.
\item[\symbol{91}group\symbol{93}] An optional string-valued field that can be used as a name to group together a set of records. Typical uses include to group the introns and exons in one gene prediction (or experimentally verified gene structure). In Version 2, group must be defined within a Tag-Value pair. Tags must be standard identifiers (\symbol{91}A-Za-z\symbol{93}\symbol{91}A-Za-z0-9\_\symbol{93}$\ast$). Free text values must be quoted within double quotes. Examples for group field can be found in table~\ref{GFFgroups}. Standard table for Group Tag Identifiers has not yet been completely formalized, however a useful constraint is that they are equivalent, where appropriate, to DDBJ/EMBL/GenBank feature `qualifiers' of given features\footnote{See the EMBL feature and qualifiers description at:\\\centerline{\bfseries\texttt{http://www3.ebi.ac.uk/Services/WebFeat/}}}.
\end{description}

\newlength{\wdth}
\newcommand{\defbox}[1]{\settowidth{\wdth}{#1}}
\newcommand{\texbox}[2]{\makebox[\wdth][#1]{#2}}
\newcommand{\frmbox}[1]{\framebox[\wdth]{#1}}
%
\begin{center}
\begin{table}[!ht]
\caption{Grouping GFF records.}\vspace{2ex}
\label{GFFgroups}
\setlength{\parindent}{-0.05\linewidth}
\begin{mypsframe}
\begin{minipage}[!ht][][c]{1.1\linewidth}
{\footnotesize\ttfamily
\centerline{\normalsize\bfseries Simple Group Names (GFF Version 1)}\vspace{1ex}
\begin{tabular}{llllllllll}
\hline\hline
\defbox{dJ102G20}\texbox{l}{CETBB} &
\defbox{GD\_mRNA}\texbox{l}{search} &
\defbox{similarity}\texbox{l}{cds} &
\defbox{32727}\texbox{l}{2189} &
\defbox{32740}\texbox{l}{2884} &
\defbox{1.6e-23}\texbox{l}{.} &
\defbox{+}\texbox{l}{+} &
\defbox{2}\texbox{l}{.} &
\defbox{Target "HBA\_HUMAN" 11 55}\texbox{l}{125}\\
rt2202 &predict &gene &1289 &12852 &64.07 &- &. &trypsin \\
MMPROT &blast &similarity &32727 &32740 &1.6e-23 &+ &1 &"RNA polymerases" \\
\end{tabular}\\[2.5ex]
\centerline{\normalsize\bfseries Tag-Value Group Names (GFF Version 2)}\vspace{1ex}
\begin{tabular}{llllllllll}
\hline\hline
\defbox{dJ102G20}\texbox{l}{jj\_lk2} &
\defbox{GD\_mRNA}\texbox{l}{finder} &
\defbox{similarity}\texbox{l}{cds} &
\defbox{32727}\texbox{l}{6718} &
\defbox{32740}\texbox{l}{7051} &
\defbox{1.6e-23}\texbox{l}{.} &
\defbox{+}\texbox{l}{-} &
\defbox{2}\texbox{l}{.} &
\defbox{Target "HBA\_HUMAN" 11 55}\texbox{l}{Transcript "1"}\\
dJ102G20 &GD\_mRNA &exon &7105 &7201 &. &- &2 &Sequence "dJ102G20.C1.1" \\
seq1 &BLASTX &similarity &101 &235 &87.1 &+ &0 &Target "HBA\_HUMAN" 11 55\\
\end{tabular}}
\end{minipage}
\end{mypsframe}
\end{table}
\end{center}

\begin{description}
\item[Comments] Comments are allowed starting with character `\#', everything following `\#' until the end of the line is ignored. Effectively this can be used in two ways: at the beginning of the line to make the whole line a comment, or the comment could come after all the required fields on the line.  
\item[Meta Information] You can define optionally a number of special comment lines for meta information at the top of your gff file with `\#\#'. Current proposed `\#\#' lines are: 
\begin{description}
\item[\small\texttt{\#\#gff-version \{version}\}]\ \\GFF version, current version is 2.
\item[\small\texttt{\#\#source-version \{source\} \{version\}}]\ \\You can record program or package version generated the data in this file.
\item[\small\texttt{\#\#date \{date\}}]\ \\The date the file was made, or perhaps when prediction programs were run. Use of astronomical format is recommended (1997-11-08 for 8th November 1997), first because this sort properly, and second to avoid any US/European bias. 
\item[\parbox{0.5\linewidth}{\small\ttfamily\#\#DNA \{seqname\}\\\#\#acggctcggattggcgctggatgatagatcagacgac\\\#\#...\\\#\#end-DNA}]\ \\To give a DNA sequence. Several people have pointed out that it may be convenient to include the sequence in the file. It should not become mandatory to do so. Often the seqname will be a well-known identifier, and the sequence can easily be retrieved from a database, or an accompanying file. 
\item[\small\texttt{\#\#sequence-region \{seqname\} \{start\} \{end\}}]\ \\To indicate that this file only contains entries for the specified subregion of a sequence. 
\end{description}
\end{description}\vspace{-1ex}

\begin{table}[!ht]
\caption{Some remarks on GFF Standard Format Version 2.}
\label{GFFremarks}
\begin{center}
\begin{mypsframe}
\begin{minipage}[!ht][][c]{0.8\linewidth}
  \begin{itemize}\setlength{\itemsep}{0ex plus0.1ex}
    \item[$\triangleright$] Intended to be easy to parse and process.
    \item[$\triangleright$] Field separator must be a TAB character (`$\backslash$t').
    \item[$\triangleright$] Fields must not include whitespace.
    \item[$\triangleright$] $<$start$>$ must be lower or equal than $<$end$>$.
    \item[$\triangleright$] When there is no $<$score$>$ you should use `.'.
    \item[$\triangleright$] When $<$strand$>$ is not relevant you should use `.'.
    \item[$\triangleright$] Available $<$frames$>$ are `.', `0', `1' and `2'.
    \item[$\triangleright$] Tag-Value pairs accepted for $<$group$>$ field.
	\begin{itemize}
    	\item[$\circ$] Group tag must be a standard identifier (\symbol{91}A-Za-z\symbol{93}\symbol{91}A-Za-z0-9\_\symbol{93}$\ast$).
    	\item[$\circ$] Free text values must be quoted within double quotes.
	\end{itemize}
  \end{itemize}
\end{minipage}
\end{mypsframe}
\end{center}
\end{table}



	\sctn{Records structure for standard GFF format}

	%	\input{GFFstandard} % GFF version 1 and 2, GFF aln, GFF vector

	\sctn{GFF format derivatives for alignments}

	%	\input{GFFstandard} % APLOT version '2'

	\sctn{{\prog} philosophy}

	%	%
% Philosophy_gff2ps.tex
%
% Main points on ideas behind ``gff2ps''.
%
% $Id: Philosophy_gff2ps.tex,v 1.1 2002-12-24 13:07:37 jabril Exp $
%

The programming philosophy underlying \prog\ can be summarized onto these points:
%\begin{minipage}[b][16cm][c]{14cm}

\begin{figure}[!ht]
\begin{center}
	\input{Blocks}
	\caption[Plot distribution for elements found in GFF records]{Here, you can observe in upper Block how strands are distributed on block area, and also how sources have reverse order in reverse strand. You can disable visualization for any strand and place one or more blocks per page. Imagine that lower Block has the same distribution as upper one, but the figure is only showing one zoomed strand. That strand also focuses on one source track, what is represented on it, and how we represent grouped and un-grouped elements generated from GFF-features. There is also shown overlapped elements and groups, each of them can be treated in different ways by \prog.}
	\label{plothierarchy} 
\end{center}
\end{figure}

%
%\hspace{-2.5cm}
\begin{figure}[!ht]
\begin{center}
\fbox{
\begin{minipage}[!ht][][c]{\linewidth}\scriptsize\ttfamily
\begin{center}
\begin{tabular}{ccc}
\textbf{\normalsize GFF Pattern}                  &               & \textbf{\normalsize Group Examples} \\ \hline\hline
\\
\symbol{91}1--8\symbol{93} .$\ast$  \symbol{91}\ldots\symbol{93}   & $\Rightarrow$ & $\left\{\begin{array}{c}\mbox{123}\\\mbox{DMSELE.1}\\\mbox{Clone\_33223}\end{array}\right.$ \\
\\
\symbol{91}1--8\symbol{93} ".$\ast$" \symbol{91}\ldots\symbol{93}  & $\Rightarrow$ & $\left\{\begin{array}{c}\mbox{"123"}\\\mbox{"DMSELE.1"}\\\mbox{"Clone\_33223"}\\\mbox{"Brain K+ Channel"}\end{array}\right.$ \\
\\
\symbol{91}1--8\symbol{93} \symbol{91}A-Za-z\symbol{93}\symbol{91}A-Za-z0-9\_\symbol{93}$\ast$ ".$\ast$" \symbol{91}\ldots\symbol{93} & $\Rightarrow$ & $\left\{\begin{array}{c}\mbox{target "123"}\\\mbox{SIMILARITY "DMSELE.1"}\\\mbox{label "Clone\_33223"}\\\mbox{Putative\_Protein "Brain K+ Channel"}\end{array}\right.$ \\
\end{tabular}
\end{center}
\end{minipage}}
\noindent
\caption[Group formats read by \prog.]{Group formats read by \prog. `\texttt{\symbol{91}1--8\symbol{93}}' represents the first eight gff-fields of each GFF-record. `\symbol{91}\ldots\symbol{93}' corresponds to extra fields that are not used by our program.}
\label{valid_groups}
\end{center}
\end{figure}
%

\begin{itemize}
%\setlength{\parsep}{0ex plus0ex}
%\setlength{\itemsep}{0ex plus0ex}
%\setlength{\topsep}{0ex plus0ex}
%\setlength{\partopsep}{0ex plus0ex}
\item[$\bullet$] We want to generate comprehensive plots of all `GFF-able' features in order to compare genomic sequences from different sources. Although developed initially to display features annotated from different sources on a single sequence, it can also be used for displaying annotations from one (or more) sources on a number of sequences. This can be useful, for instance to compare the genomic structure of different sequences.
\item[$\bullet$] \prog\ can parse Version 1 and Version 2 GFF-formatted records, records that are not compliant with GFF-format are discarded warning user afterwards. Field separator and group field were defined slightly different from one version to the other, here it is explained how \prog\ deals with those differences.
If records from input GFF-files contain tabulators as field separator program assumes that record is GFF Version 2 formatted, else if it finds blank-spaces as field separator then switches to Version 1.\\
Groups must be defined as a `tag-value' pair in GFF Version 2, where `tag' must be a standard identifier (\symbol{91}A-Za-z\symbol{93}\symbol{91}A-Za-z0-9\_\symbol{93}$\ast$) and `value' must be a free-text string enclosed between double-quotes (`\,\texttt{"}.$\ast$\texttt{"}\,'), for example `\texttt{target "HS new gene"}' fits that group format. Other group formats force program to switch to GFF Version 1: if there are more than nine fields it tries to find the `tag-value' pattern, else assumes that the ninth field is the group name as a quoted or not free-text string. See figure~\ref{valid_groups} for group formats than can be processed by \prog.\\
%\item[$\bullet$] \prog\ Works with GFF Version 2 formatted lines, that implies that records are checked in a slightly different way to Version 1, mainly due to field separator constraint and group field definition. Whereas the group can be used to set groups on the plot, we recommend to define it as a label not as a fixed tag (we are working for `target' and `transcript' following labels will be also used for grouping the gff elements), see figure~\ref{GFFgroups} on section~\ref{sec:philoGFF} for some grouping examples. Records that are not compliant with GFF-format are discarded, warning user for that and not used by \prog.
\item[$\bullet$] The program must be easy to use, all its parameters are set by default inside the program. But must be easy to change plot options which can be modified by a default custom file, also by a working custom file (smaller than default file and provided to introduce small changes for single plots), and some of them from command-line. The main goal is to define a system in which can be easily added new options or redefine old ones. Another issue is that configuration files are plain-ASCII text, so they can be edited with small and simple text editors. To know more about this issue you can read section~\ref{sec:VdefCF} and~\ref{sec:CustFiles}. The program can work in background or used in a \textsc{Unix} pipeline working as a filter (section~\ref{sec:unixCL}).
\item[$\bullet$] Source order from input gff-file is preserved when reading those files, it means that you can easily switch source order in your plot swapping the order of the input files. Sources are shown in plots giving a mirroring symmetry axes for strands forward and reverse, so forward sources are shown by its ordering from top to bottom and reverse sources are shown counter-wise, while records without a defined strand are placed in the plot area between the two strands areas.
\item[$\bullet$] User-defined custom files can handle regular expressions, allowing us to define any attribute variable for multiple similar features ---in GFF-elements, group or source blocks--- in one line, to know about this feature you can look at section~\ref{sec:UsingREGEXP}.
\item[$\bullet$] Some of the previous items leads to hierarchical plots, in which Pages are the highest element, and Blocks are defined within them ---this feature allows you to get multiple vertical and/or horizontal pages, also pages with multiple blocks---. Inside Blocks may appear any Strand ---forward, reverse, or no-frame (defined in GFF-records as `.')---. Each Strand presents each Source as one plot line or more ---if you want to display overlapping groups in the same line and/or in different lines or if you prefer to split data-sets between grouped and un-grouped features---, meanwhile Groups belonging to them define features for displaying sets of GFF-features, which are the basic plot elements (schemed in figure~\ref{plothierarchy}). Page number and Blocks per page are set as a Layout variables (see section~\ref{sec:layoutft}), whereas the rest of other elements are defined in specific fields from GFF-files records: Sources are named at second field, GFF-features came from third field, Strands from field seven, and Groups from ninth (see section~\ref{sec:GFFdefs} and tables~\ref{GFFrecords} and \ref{GFFgroups}). Start, End, Score and Frame are defined into GFF-features as plot attributes. 
\item[$\bullet$] You can switch on/off visualization of overlapping groups in several lines for same source track ---option by default---, or you can fit them into a single track. \prog\ minimizes the number of lines needed for that when displaying overlapping groups in different lines. Individual non-grouped GFF-elements are treated as a one element group, which allows you to display also individual elements without overlapping. If you choose to print in a single line, you can also define layers for each set of overlapping GFF-features, maybe `exons' at top and `cds' at bottom, enhancing viewing for any of the elements. See sections~\ref{sec:gffelemft} and~\ref{sec:groupft}.
\item[$\bullet$] You must remember that \prog\ converts all upper-case characters for \textbf{features} to lower-case. In order to prevent that one user had defined `Exon', another `exon' and other `EXON', our program convert them to `exon'.
\item[$\bullet$] Scores control feature width, but in order to prevent problems when working with data-sets provided by different programs they are re-scaled for each source, using maximum and minimum values as a score range within all scores are re-calculated. Default score is set to maximum value for each source (when parsing gff-files and a record contains a `.' in Score field). Further information can be found in section~\ref{sec:sourceft}.
\item[$\bullet$] We have defined three block areas where are displayed Strands, you can switch on/off any of them and visualize one up to three strands. Forward strand (`+') is always shown at upper area, while reverse strand is always shown at lower area. When strand is not defined (`.') elements are placed in the central area, between forward and reverse areas. Source order is preserved from input files, in forward strand sources follows that ordering up to down, also in the no-strand area, but in reverse strand are shown down to up, so you have a horizontal symmetry axes between forward/no-strand and reverse strand.
\item[$\bullet$] Features for which frame is specified are plotted using a two color code schema. The upstream half of the graphical element representing the frame of feature and the downstream half the complement modulus three of its remainder. This is useful to check frame consistency between adjacent features (for instance, predicted exons). Two adjacent features are frame-compatible when the color of the downstream half of the upstream feature matches the color of the upstream half of the downstream feature. This two-color code schema, however, is only meaningful when the frame has been defined relative to the feature, and not relative to the sequence. We have defined four independent frames in \prog (`.', `0', `1', and `2'), that is used by a coloring procedure for visualizing frame and remainder within shapes. Colors for frames are also independently defined from feature color definition, and can be customized too. The complement modulus three for ``remainder'' is calculated by our program following this formula:\\[1ex]
\centerline{\label{fml:remainder}$\mbox{\textsl{``Remainder''}} = ( 3 - ( \mbox{\textsl{End}} - ( \mbox{\textsl{Start}} + \mbox{\textsl{Frame}} ) + 1 ) \bmod 3 ) \bmod 3$}
\item[$\bullet$] Score vectors are shown as score-dependent color gradients and are read from one line which contain such vectors in those fields beyond group field. We will implement in next version procedures for showing them as continuous or discrete functions, spikes, and so on... You must use the following format for the group and the following fields (next version will work also with multiple GFF score-feature single-records) :\label{vectordef}\\

\centerline{\small\symbol{91}Fields 1 to 8\symbol{93}\ \ Tag\ ``Value''\ \ \textbf{score};\ \ \textbf{Window} \textit{window};\ \ \textbf{Step} \textit{step};\ \ \textbf{Scores} \textit{score} \symbol{91}\ldots\symbol{93} \textit{score}}

\end{itemize}







%
% %
% Settings.tex
%
% How to install gff2ps for ``gff2ps Manual''.
%
% $Id: Settings.tex,v 1.2 2003-03-04 18:55:22 jabril Exp $
%
%\newcommand{\csh}[1]{\texttt{\textbf{[\textsl{cshell}]\$} #1}}
%\newcommand{\bsh}[1]{\texttt{\textbf{[\textsl{bshell}]\$} #1}}
%\newcommand{\pgm}[1]{\textit{#1}}

\chap{Using {\prog}}
	
In this chapter it is explained how you can set system variables in order to start working with \prog. This program was designed to work under UNIX and has been tested under Irix, Solaris and Linux. In table~\ref{testedon} you can see the program versions with which we have worked. \prog\ has three inner modules: the shell script, the GNU awk script, and the PostScript prologue code. This prologue contains all procedure sets we have written to obtain the PostScript plots and it is embedded into `\texttt{gff2ps}'. This is a Bourne shell script ---using \textit{sh} or \textit{bash} (systems under Linux have a link for \textit{sh} to \textit{bash})--, that handles with command-line options, checks if given files exist and pass them to the GNU awk script. This loads data records and custom definitions generating the full PostScript output. You can visualize that with a PostScript viewer ---like ghostview, xpsview--- or send to a PostScript printer to obtain a hardcopy.\par

The main difference from older versions is that GNU awk script is now included within the shell script to facilitate installation of our program.\par

\sctn{Installation}
%	\input{Install}

\sctn{Command-Line}
%	\input{CommandLine}

\sctn{Reporting Bugs}
%	\input{BugReport}


%
% %
% Options.tex
%
% Understanding customization parameters.
%
% $Id: Options.tex,v 1.1 2003-03-04 18:55:21 jabril Exp $

\chap{Customizing {\prog} Output}

%\newpage

\sctn{Command-line Options Description}

% This section describes command-line options and customization variables as they will appear in the {\prog} manual.
\newcommand{\setlist}{
 \setlength{\leftmargin}{0.5cm}
 \setlength{\itemindent}{-0.5cm}
 \setlength{\labelwidth}{0pt}
 \setlength{\labelsep}{0pt}
 \renewcommand{\labelitemi}{}
} %setlist

 \begin{itemize}\setlist
  \input tables/AplotCLopts_dsc.tex % tables/AplotCLoptions.tex
 \end{itemize}

\newpage

\sctn{Customization Variables Description}

% tables/AplotCustomVars.tex

\subsctn{Layout Attributes}

 \begin{itemize}\setlist
  \input tables/AplotCFvars_dscLayout.tex
 \end{itemize}

\subsctn{GFF-Feature Attributes}

 \begin{itemize}\setlist
  \input tables/AplotCFvars_dscFeature.tex
  \item 
 \end{itemize}

\subsctn{GFF-Group Attributes}

 \begin{itemize}\setlist
  \input tables/AplotCFvars_dscGroup.tex
  \item 
 \end{itemize}

\subsctn{GFF-Strand Attributes}

 \begin{itemize}\setlist
  \input tables/AplotCFvars_dscStrand.tex
  \item 
 \end{itemize}

\subsctn{GFF-Source Attributes}

 \begin{itemize}\setlist
  \input tables/AplotCFvars_dscSource.tex
  \item 
 \end{itemize}

\subsctn{GFF-Sequence Attributes}

 \begin{itemize}\setlist
  \input tables/AplotCFvars_dscSequence.tex
  \item 
 \end{itemize}

\subsctn{Special Customization Features}

 \begin{itemize}\setlist
  \input tables/AplotCFvars_dscExtra.tex
  \item 
 \end{itemize}



%
% \include{GettingStart.tex}

%
% \include{Appendix.tex}

%\newpage

\sctn{Command-line Options Description}

% This section describes command-line options and customization variables as they will appear in the {\prog} manual.
\newcommand{\setlist}{
 \setlength{\leftmargin}{0.5cm}
 \setlength{\itemindent}{-0.5cm}
 \setlength{\labelwidth}{0pt}
 \setlength{\labelsep}{0pt}
 \renewcommand{\labelitemi}{}
} %setlist

 \begin{itemize}\setlist
  \input tables/AplotCLopts_dsc.tex % tables/AplotCLoptions.tex
 \end{itemize}

\newpage

\sctn{Customization Variables Description}

% tables/AplotCustomVars.tex

\subsctn{Layout Attributes}

 \begin{itemize}\setlist
  \input tables/AplotCFvars_dscLayout.tex
 \end{itemize}

\subsctn{GFF-Feature Attributes}

 \begin{itemize}\setlist
  \input tables/AplotCFvars_dscFeature.tex
  \item 
 \end{itemize}

\subsctn{GFF-Group Attributes}

 \begin{itemize}\setlist
  \input tables/AplotCFvars_dscGroup.tex
  \item 
 \end{itemize}

\subsctn{GFF-Strand Attributes}

 \begin{itemize}\setlist
  \input tables/AplotCFvars_dscStrand.tex
  \item 
 \end{itemize}

\subsctn{GFF-Source Attributes}

 \begin{itemize}\setlist
  \input tables/AplotCFvars_dscSource.tex
  \item 
 \end{itemize}

\subsctn{GFF-Sequence Attributes}

 \begin{itemize}\setlist
  \input tables/AplotCFvars_dscSequence.tex
  \item 
 \end{itemize}


%%%%%%%%%%%%%%%%%%%% BACKMATTER: Bibliography, APPENDIXES, Index.

% \newpage
% 
% \bibliographystyle{apalike}
% \bibliography{/home1/rguigo/docs/biblio/References}

\appendix

\newpage
\fancyhead[RO,LE]{\thepage}
\fancyhead[LO,RE]{}
\fancyfoot[LO,LE]{}
\fancyfoot[RO,RE]{}
\renewcommand{\headrulewidth}{1pt}
\renewcommand{\footrulewidth}{0pt}
\geometry{noheadfoot}
%\enlargethispage*{40cm}

\sctn{Command-Line Options Summary for \prog.}
%\vskip -0.25cm

%\vfill
% 
\begin{table}[!ht]
\centering
% \begin{center}
 \label{tbl:CLopt} 
 \setlength{\fboxsep}{10pt}
 \footnotesize 
%
\fbox{
  \input tables/AplotCLopts_tbl.tex % tables/AplotCLoptionsTbl.tex
} % fbox
%
\vskip 0.25cm
%
\fbox{
\begin{tabular}{rl}
\textbf{NOTE:}  &
  \textbf{ When a parameter is required, it applies for both short and long options. }\\
\pa{int}       &
  An integer value. \\
\pa{float}     &
  A float value. \\
\pa{string}    &
  A free text string, single or double-quoted \\[-0.5ex]
 &  if special chars or white-spaces/tabs are present respectively. \\
\pa{format}    &
  Page format, see available values on Appendix~\ref{tbl:PageSztbl} table. \\
\pa{color}     &
  A color name chosen from table on Appendix~\ref{tbl:CMYKcolortbl}. \\
\pa{file}      &
  A valid file name (including path if necessary). \\
\end{tabular}
} % fbox
%
%\end{center}
\vskip -2cm
\end{table}
%\vfill

\newpage

\sctn{Customization Variables Sumary for \prog.}

% tables/AplotCustomVarsTbl.tex

\newpage
\landscape
\begin{center}
\begin{footnotesize}
\setlength{\parindent}{-0.45cm}

  \input tables/AplotCFvars_tblplayout.tex
  \vskip 1cm
  \input tables/AplotCFvars_tblzoom.tex
\newpage
  \input tables/AplotCFvars_tbllabel.tex
\newpage
  \input tables/AplotCFvars_tblticks.tex
  \vskip 1cm
  \input tables/AplotCFvars_tblaplot.tex
  \vskip 1cm
  \input tables/AplotCFvars_tblgeneral.tex

\newpage
% \vskip 1cm

  \input tables/AplotCFvars_tblfeat.tex
  \vskip 1cm

  \input tables/AplotCFvars_tblgrp.tex
  \vskip 1cm

  \input tables/AplotCFvars_tblstr.tex
  \vskip 1cm

  \input tables/AplotCFvars_tblsrc.tex
  \vskip 1cm

  \input tables/AplotCFvars_tblseq.tex

\end{footnotesize}
\end{center}
\newpage
\endlandscape

\sctn{CMYK color definition}

\input tables/AplotColorTbl.tex

\newpage

\sctn{Page format definition}

\input tables/AplotPageSizeTbl.tex

\end{document}



