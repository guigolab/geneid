%
% AplotCFvars_dscGroup.tex
%
% Description items for GROUP
%
% $Id: AplotCFvars_dscGroup.tex,v 1.1 2002-12-24 13:07:37 jabril Exp $ 
%
\idesc{\op{hide} $\,=\,$ \pp{boolean} \hfill [ \vp{off} ]}
   { {\tbdef} }
%
\idesc{\op{group\_color} $\,=\,$ \pp{color} \hfill [ \vp{fg} ]}
   { {\tbdef} }
%
\idesc{\op{group\_shape} $\,=\,$ \pp{group\_shape} \hfill [ \vp{bracket} ]}
   { {\tbdef} }
%
\idesc{\op{show\_group\_limits} $\,=\,$ \pp{boolean} \hfill [ \vp{off} ]}
   { {\tbdef} }
%
\idesc{\op{group\_label} $\,=\,$ \pp{string} \hfill [ \bydef ]}
   { {\tbdef} }
%
\idesc{\op{show\_group\_label} $\,=\,$ \pp{boolean} \hfill [ \vp{on} ]}
   { {\tbdef} }
%
\idesc{\op{group\_layer} $\,=\,$ \pp{integer} \hfill [ \bydef ]}
   { Setting layer on annotation tiers where to draw this group shape. By default all groups are drawn on layer 0, which represents the bottomest layer. Higher layer values move features forward, this allows users to visualize groups that were hidden by others if they were overlapping. {\prog} sorts groups prior visualization by acceptor, but before that, it sorts by layer, so groups having smaller acceptor coords and are longer than others can be forced to appear on top. This is done before sorting by layer all the features, so that, features for the choosen group are raised to the group layer. This means that you actually have two layer levels, the group and the feature layer levels. 
 
 Imagine you have ``exons'' and the ``mrna'' features for a gene, and would like to display the ``mrna'' (which contains all the ``exons'' between its starting and final coordinates), as a lower ---than the ``exons''--- box. If all feature had same layer you will see fragments of the ``mrna'' rising between each consecutive ``exons''. Increasing layer number for ``mrna'' while preserving layer number for ``exons'' will make ``mrna'' box to be fully drawn on top of the exons. }
%
\idesc{\op{feature\_color} $\,=\,$ \pp{color} \hfill [ \bydef ]}
   { {\tbdef} }
%
\idesc{\op{ribbon\_color} $\,=\,$ \pp{color} \hfill [ \bydef ]}
   { {\tbdef} }
%
