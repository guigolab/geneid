\idesc{\shortstack[l]{\ \op{-h}  \\ \op{-\/-help} } \hfill }
   { Shows command-line help. }
%
\idesc{\op{-\/-version}  \hfill }
   { Shows current program version and exits. }
%
\idesc{\shortstack[l]{\ \op{-v}  \\ \op{-\/-verbose} } \hfill }
   { By default, warnings and errors are sent to standard error. This option switches on process reporting messages to appear on standard error too. }
%
\idesc{\shortstack[l]{\ \op{-V} \pp{filename} \\ \op{-\/-logs-filename} \pp{filename}} \hfill }
   { If is possible to open \pp{filename} then such file will contain all the messages and warnings produced by the program even though they were disabled from standard error with \op{quiet} option. }
%
\idesc{\shortstack[l]{\ \op{-q}  \\ \op{-\/-quiet} } \hfill }
   { This option disables any message or warning from standard error. It does not disable error report because such errors are pointing a problem which can make {\prog} produce unexpected results. Solve the cause of such problems before continuing to run the program in a silent mode. }
%
\idesc{\shortstack[l]{\ \op{-P} \pp{width,height} \\ \op{-\/-page-bbox} \pp{width,height}} \hfill \op{page\_bbox}=\pp{width,height}}
   { This option allows users to define an arbitrary page format, by defining the width and height for that page ({\prog} assigns automatically a format name for this new page dimensions). See \op{page\_bbox} customization variable on page~\pageref{sec:pagebbox} (section~\ref{sec:pagebbox}) for further info on this command-line option. It overrides any \op{page-size} command-line option (and any related variable passed through custom files). \pp{width} and \pp{height} are set to points if no unit is given, now you can use pt, mm, cm or in. }
%
\idesc{\shortstack[l]{\ \op{-p} \pp{format\_name} \\ \op{-\/-page-size} \pp{format\_name}} \hfill \op{page\_size}=\pp{format\_name}}
   { \pp{format\_name} is a pre-defined page format from table~\ref{tbl:PageSizes}. By default is set to \vp{a4}. }
%
\ijoin{\op{-\/-margin-left} \pp{length} \hfill \op{margin\_left}=\pp{length}}
%
\ijoin{\op{-\/-margin-right} \pp{length} \hfill \op{margin\_right}=\pp{length}}
%
\ijoin{\op{-\/-margin-top} \pp{length} \hfill \op{margin\_top}=\pp{length}}
%
\idesc{\op{-\/-margin-bottom} \pp{length} \hfill \op{margin\_bottom}=\pp{length}}
   { You can set page margin with those four variables. \pp{length} can be given in points, milimeters, centimeters or inches (pt, mm, cm or in, respectively), but is set to points if no units are provided. }
%
\ijoin{\shortstack[l]{\ \op{-B} \pp{color} \\ \op{-\/-background-color} \pp{color}} \hfill \op{background\_color}=\pp{color}}
%
\idesc{\shortstack[l]{\ \op{-F} \pp{color} \\ \op{-\/-foreground-color} \pp{color}} \hfill \op{foreground\_color}=\pp{color}}
   { You can change background ---say here page filling--- and foreground ---say here text, outlines and tickmarks--- color for the page. Available colors are defined on table~\ref{tbl:CMYKcolor}, default values are \vp{white} and \vp{black}, for background and foreground respectively. }
%
\idesc{\shortstack[l]{\ \op{-T} \pp{string} \\ \op{-\/-title} \pp{string}} \hfill \op{title}=\pp{string}}
   { Setting the main title to \pp{string} for the current figure, by default showing \vp{align\_name} in the form of \vp{sequence1\_name} x \vp{sequence2\_name}. }
%
\idesc{\shortstack[l]{\ \op{-t} \pp{string} \\ \op{-\/-subtitle} \pp{string}} \hfill \op{subtitle}=\pp{string}}
   { Setting \pp{string} as the current figure subtitle, by default an empty string. }
%
\idesc{\shortstack[l]{\ \op{-X} \pp{string} \\ \op{-\/-x-label} \pp{string}} \hfill \op{x\_label}=\pp{string}}
   { {\tbdef} }
%
\idesc{\shortstack[l]{\ \op{-Y} \pp{string} \\ \op{-\/-y-label} \pp{string}} \hfill \op{y\_label}=\pp{string}}
   { {\tbdef} }
%
\idesc{\shortstack[l]{\ \op{-L} \pp{string} \\ \op{-\/-percent-box-label} \pp{string}} \hfill \op{percent\_box\_label}=\pp{string}}
   { {\tbdef} }
%
\idesc{\shortstack[l]{\ \op{-l} \pp{string} \\ \op{-\/-extra-box-label} \pp{string}} \hfill \op{extra\_box\_label}=\pp{string}}
   { {\tbdef} }
%
\idesc{\shortstack[l]{\ \op{-x} \pp{pos..pos} \\ \op{-\/-x-sequence-coords} \pp{pos..pos}} \hfill \op{x\_sequence\_coords}=\pp{pos..pos}}
   { {\tbdef} }
%
\idesc{\shortstack[l]{\ \op{-S} \pp{pos} \\ \op{-\/-start-x-sequence} \pp{pos}} \hfill \op{x\_sequence\_start}=\pp{pos}}
   { {\tbdef} }
%
\idesc{\shortstack[l]{\ \op{-E} \pp{pos} \\ \op{-\/-end-x-sequence} \pp{pos}} \hfill \op{x\_sequence\_end}=\pp{pos}}
   { {\tbdef} }
%
\idesc{\shortstack[l]{\ \op{-y} \pp{pos..pos} \\ \op{-\/-y-sequence-coords} \pp{pos..pos}} \hfill \op{y\_sequence\_coords}=\pp{pos..pos}}
   { {\tbdef} }
%
\idesc{\shortstack[l]{\ \op{-s} \pp{pos} \\ \op{-\/-start-y-sequence} \pp{pos}} \hfill \op{y\_sequence\_start}=\pp{pos}}
   { {\tbdef} }
%
\idesc{\shortstack[l]{\ \op{-e} \pp{pos} \\ \op{-\/-end-y-sequence} \pp{pos}} \hfill \op{y\_sequence\_end}=\pp{pos}}
   { {\tbdef} }
%
\idesc{\op{-\/-x-sequence-zoom} \pp{pos..pos} \hfill \op{x\_sequence\_zoom}=\pp{pos..pos}}
   { {\tbdef} }
%
\idesc{\op{-\/-y-sequence-zoom} \pp{pos..pos} \hfill \op{y\_sequence\_zoom}=\pp{pos..pos}}
   { {\tbdef} }
%
\idesc{\shortstack[l]{\ \op{-Z}  \\ \op{-\/-zoom} } \hfill \op{zoom}=\vp{on}}
   { {\tbdef} }
%
\idesc{\shortstack[l]{\ \op{-z}  \\ \op{-\/-zoom-area} } \hfill \op{zoom\_area}=\vp{on}}
   { {\tbdef} }
%
\idesc{\shortstack[l]{\ \op{-A} \pp{seqXname:seqYname} \\ \op{-\/-alignment-name} \pp{seqXname:seqYname}} \hfill \op{alignment\_name}=\pp{seqXname:seqYname}}
   { When you are providing several alignments from input, you can select which one to be plotted. By default program uses first alignment found in the input stream if \op{x\_sequence\_name} and \op{y\_sequence\_name} were also not defined, else it will try combining those variables if they are set by user or relying on their default values. See \op{alignment\_name} customization variable description on section~\ref{sec:seqXseqY}, page~\pageref{sec:seqXseqY}, for further info. It is also explained there but remember that you should take care of unexpected side effects when setting different sequence names to this command-line option and its siblings, \op{x-sequence-name} and \op{y\_sequence\_name}, if you set one of them in a custom file and the others on command-line or viceversa. }
%
\idesc{\shortstack[l]{\ \op{-N} \pp{seqXname} \\ \op{-\/-x-sequence-name} \pp{seqXname}} \hfill \op{x\_sequence\_name}=\pp{seqXname}}
   { You can choose the sequence that is going to be drawn along X-axis, or what is equivalent, which annotation will appear on that axis. See \op{alignment-name} description to know more about these variables. }
%
\idesc{\shortstack[l]{\ \op{-n} \pp{seqYname} \\ \op{-\/-y-sequence-name} \pp{seqYname}} \hfill \op{y\_sequence\_name}=\pp{seqYname}}
   { Here you will define which sequence is drawn on Y-axis. Take a look to \op{alignment-name} to get more info about the interactions between these variables. }
%
\idesc{\shortstack[l]{\ \op{-r}  \\ \op{-\/-aplot-xy-noteq} } \hfill \op{aplot\_xy\_same\_length}=\vp{off}}
   { {\tbdef} }
%
\idesc{\shortstack[l]{\ \op{-R} \pp{X/Y ratio} \\ \op{-\/-xy-axes-scale} \pp{X/Y ratio}} \hfill \op{aplot\_xy\_scale}=\pp{X/Y ratio}}
   { {\tbdef} }
%
\idesc{\shortstack[l]{\ \op{-W}  \\ \op{-\/-aln-scale-width} } \hfill \op{alignment\_scale\_width}=\vp{on}}
   { {\tbdef} }
%
\idesc{\shortstack[l]{\ \op{-w}  \\ \op{-\/-aln-scale-color} } \hfill \op{alignment\_scale\_color}=\vp{on}}
   { {\tbdef} }
%
\idesc{\shortstack[l]{\ \op{-K} \pp{ribbon-type} \\ \op{-\/-show-ribbons} \pp{ribbon-type}} \hfill \op{show\_ribbons}=\pp{ribbon-type}}
   { {\tbdef} }
%
\idesc{\shortstack[l]{\ \op{-G}  \\ \op{-\/-show-grid} } \hfill \op{show\_grid}=\vp{on}}
   { {\tbdef} }
%
\idesc{\shortstack[l]{\ \op{-I}  \\ \op{-\/-show-percent-box} } \hfill \op{show\_percent\_box}=\vp{on}}
   { {\tbdef} }
%
\idesc{\shortstack[l]{\ \op{-i}  \\ \op{-\/-hide-percent-box} } \hfill \op{show\_percent\_box}=\vp{off}}
   { {\tbdef} }
%
\idesc{\shortstack[l]{\ \op{-O}  \\ \op{-\/-show-extra-box} } \hfill \op{show\_extra\_box}=\pp{on}}
   { {\tbdef} }
%
\idesc{\shortstack[l]{\ \op{-o}  \\ \op{-\/-hide-extra-box} } \hfill \op{show\_extra\_box}=\pp{off}}
   { {\tbdef} }
%
\idesc{\shortstack[l]{\ \op{-D} \pp{color} \\ \op{-\/-aplot-box-color} \pp{color}} \hfill \op{aplot\_box\_bgcolor}=\pp{color}}
   { {\tbdef} }
%
\idesc{\shortstack[l]{\ \op{-d} \pp{color} \\ \op{-\/-percent-box-color} \pp{color}} \hfill \op{percent\_box\_bgcolor}=\pp{color}}
   { {\tbdef} }
%
\idesc{\shortstack[l]{\ \op{-b} \pp{color} \\ \op{-\/-extra-box-color} \pp{color}} \hfill \op{extra\_box\_bgcolor}=\pp{color}}
   { {\tbdef} }
%
\idesc{\op{-\/-nopswarnings}  \hfill \op{show\_ps\_warnings}=\vp{off}}
   { When input data is missing, regardless you do not provide X or Y sequence annotation, or even alignment records, {\prog} will include a warning message in the final {\ps} output. This will help to identify what is missing from such input data, but sometimes it must be ignored and you may need to switch those warnings off. You can switch off on a single figure from command-line, or disable from a default customization file variable. The last choice it is not recommended unless you already know what you are doing and you are sure that you will get annotations along only for one of the axes or there are sequence pairs that do not have the corresponding alignment data. }
%
\idesc{\shortstack[l]{\ \op{-a}  \\ \op{-\/-hide-credits} } \hfill \op{hide\_credits}=\vp{on}}
   { {\prog} always displays a tiny copyright footer on the output {\ps} figures that users can remove using this command-line option for a single case, or the corresponding customization variable in a default customization file for all the figures. }
%
\ijoin{\op{-\/-layout-var} '\pp{variable=value}' \hfill \op{variable}=\vp{value}}
%
\ijoin{\op{-\/-sequence-var} '\pp{sequence::variable=value}' \hfill \op{sequence::variable}=\vp{value}}
%
\ijoin{\op{-\/-source-var} '\pp{source::variable=value}' \hfill \op{source::variable}=\vp{value}}
%
\ijoin{\op{-\/-strand-var} '\pp{strand::variable=value}' \hfill \op{strand::variable}=\vp{value}}
%
\ijoin{\op{-\/-group-var} '\pp{group::variable=value}' \hfill \op{group::variable}=\vp{value}}
%
\idesc{\op{-\/-feature-var} '\pp{feature::variable=value}' \hfill \op{feature::variable}=\vp{value}}
   { Loading any of feature/group/strand/source/sequence/layout customization variables from command-line. You can set several variables by repeating any of these options, i.e.:\\[1.5ex] 
 \begin{minipage}{\linewidth} \begin{center} \sffamily\small \ldots -\/-feature-var '/\^{ }cds.*/::feature\_shape=box' -\/-feature-var 'exon::feature\_color=blue' \ldots \end{center} \end{minipage}\\[1.5ex] 
 In the previous example, '\textbf{\textsf{/\^{ }cds.*/}}' is a regular expression matching any GFF-feature which name starts with '\textbf{\textsf{cds}}', setting their shape to \vp{box}. The second line sets GFF-feature color (filling the feature shape) only for those GFF-features that match exactly with '\textbf{\textsf{exon}}' (as if we passed the feature name as a regular expression, in this case as '\textbf{\textsf{/\^{ }exon\$/}}'). 
 
 These command-line options allow users to change on the fly any variable for the current {\prog} execution. It is very useful for testing dinamically customizations before including them in the custom files. But it also lets users to keep a fixed customization file shared among several runs, while making small changes among them. Take care of protecting the parameters given to those command-line options by single-quoting them, just to avoid any undesirable side effect produced by shell-substitution of wild-chars (especially when passing regular expressions to match several elements to which set the same variable). }
%
\idesc{\shortstack[l]{\ \op{-C} \pp{filename} \\ \op{-\/-custom-filename} \pp{filename}} \hfill }
   { Loading customization parameters from \pp{filename}. Now you can load several customization files by passing this option several times. The precedence is the input order in the command-line, so, for the common definitions, the last \vp{custom\_file} will override previous \vp{custom\_files} settings. If default customization file "\texttt{.gff2aplotrc}" does exist, it is the first customization that is loaded by the program. }
%
\idesc{\shortstack[l]{\ \op{-g}  \\ \op{-\/-hide-grid} } \hfill \op{show\_grid}=\vp{off}}
   { {\tbdef} }
%
