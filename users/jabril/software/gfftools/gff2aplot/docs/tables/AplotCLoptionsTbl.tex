%
% AplotCLoptionsTbl.tex
%
% Command-Line Options for "gff2aplot".
%
% # $Id: AplotCLoptionsTbl.tex,v 1.2 2001-09-06 18:33:36 jabril Exp $
%
% \newcommand{\op}[1]{\bfseries\sffamily #1}
% \newcommand{\tp}[1]{'{\bfseries\sffamily #1}'}
% \newcommand{\pr}[1]{\mdseries\sffamily\slshape $<$#1$>$}
% \newcommand{\x}{\textmd{\,,\,}}
%
\label{sec:CLoptions}
\newcommand{\rw}[2]{ \op{ #1 } & #2 \\}
%
%%%%%%%%%%%%%%%%%%%%%%%%%%%%%%%%%%%%%%%%%%%%%%%%%%%%%%%%%%%%
\vfill 
%\setlength{\intextsep}{0ex}
%\setlength{\textfloatsep}{0ex}
%\setlength{\floatsep}{0ex}
\begin{table}[!ht]
\begin{center}
\label{tbl:CLopt}
% \refstepcounter{section}\refstepcounter{table}
% \addcontentsline{lot}{section}{
%   \thesection\hspace{1em}Shell command-line options for \prog.}
% \addcontentsline{toc}{section}{
%   \thesection\hspace{1em}Shell command-line options for \prog.}
\setlength{\fboxsep}{10pt}
\footnotesize
\fbox{
  \begin{tabular}{rl}
\rw{-h{\x}-\/-help}{Shows this help.}
\rw{-\/-version}{Shows current version and exits.}
\rw{-v{\x}-\/-verbose}
  {Verbose mode, a full report is sent to standard error.}
\rw{-V{\x}-\/-logs-filename \pr{file}}
  {Report is written to a log file.}
\rw{-q{\x}-\/-quiet}
  {Quiet mode, messages/warnings disabled (only ERRORS are reported)}
\rw{-\/-feature-var '\pr{feature::variable=value}'}
  {Set a feature customization variable from command-line.}
\rw{-\/-group-var '\pr{group::variable=value}'}
  {Set a group customization variable from command-line.}
\rw{-\/-strand-var '\pr{strand::variable=value}'}
  {Set a strand customization variable from command-line.}
\rw{-\/-source-var '\pr{source::variable=value}'}
  {Set a source customization variable from command-line.}
\rw{-\/-sequence-var '\pr{sequence::variable=value}'}
  {Set a sequence customization variable from command-line.}
\rw{-\/-layout-var '\pr{variable=value}'}
  {Set a layout customization variable from command-line.}
\rw{-O{\x}-\/-custom-filename \pr{file}}
  {Read customization parameters from file.}
  \end{tabular}
} % fbox
\vspace{0.5cm}

%\vfill
\fbox{
\begin{tabular}{rl}
\pa{int}       &
  An integer value. \\
\pa{float}     &
  A float value. \\
\pa{string}    &
  A free text string, single or double-quoted \\[-0.5ex]
 &  if special chars or white-spaces/tabs are present respectively. \\
\pa{format}    &
  Page format, see available values on Appendix~\ref{tbl:PageSztbl} table. \\
\pa{color}     &
  A color name chosen from table on Appendix~\ref{tbl:CMYKcolortbl}. \\
\pa{file}      &
  A valid file name (including path if necessary). \\
NOTE:          &
  When a parameter is required, it applies for both short and long options. \\
\end{tabular}
} % fbox
%
\end{center}
\end{table}
\vfill
%%%%%%%%%%%%%%%%%%%%%%%%%%%%%%%%%%%%%%%%%%%%%%%%%%%%%%%%%%%%
